\subsubsection{AL Platform}

{\it AL Platform} --- платформа для быстрой разработки гибких решений со сложным интерфейсом пользователя~\cite{alplatform-website}. {\it AL Platform} вводит концепцию абстрактного рабочего стола, и определяет новую методологию разработки --- <<Инструментально-ориентированное подход>>. Инструмент --- это независимый программный модуль, плагин, который размещается внутри других приложений. Идея абстрактного рабочего стола позволяет таким инструментам <<наполнять>> рабочее пространство приложения и его меню без перекомпиляции приложения или инструмента. {\it AL Platform} не является внешним самостоятельным приложением, это библиотека классов, поставляемая в виде набора {\tt .NET} сборок. 

{\it AL Platform} позволяет выстроить архитектуру приложения таким образом, чтобы оно состояло из интегрированных модулей (плагинов), которые в терминологии платформы называются инструментами. Каждый инструмент включает в себя некоторую бизнес-логику. В зависимости от решаемых задач, инструмент может использоваться в различных приложениях. Инструменты разрабатываются независимо различными командами разработчиков. Плагины могут быть собраны в разные {\tt .NET} сборки и размещены физически на разных компьютерах. {\it AL Platform} предоставляет возможность таким инструментам взаимодействовать друг с другом, минимизируя при этом накладные расходы.

С точки зрения программиста, для создания нового инструмента нужно разработать класс, унаследованный от {\tt Al.Application.Tool} или {\tt Al.DesktopApplication.DesktopTool}. Компонент приложения, отвечающий за работу с плагинами, называется менеджером инструментов. Менеджер инструментов определяет, как тот или иной инструмент будет отображаться в рабочем пространстве приложения, а также предоставляет слой инфраструктуры, необходимый для взаимодействия инструмента и приложения. 

Все возможности приложения на базе {\it AL Platform} конфигурируются с помощью {\tt XML} файла с настройками. За счёт этого достигается максимальная гибкость персонализации существующих приложений.

Важной особенностью платформы является возможность <<связывать>> инструменты между собой. Это означает, что инструмент может быть запушен в одном менеджере инструментов, а использован в другом.