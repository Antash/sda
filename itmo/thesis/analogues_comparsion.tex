\section{Сравнение с аналогами}

Единственным непосредственным аналогом разработанной платформы является VSTA. (речь идет о ПО для платформы .NET)  Сравнение будет происходить основываясь на опыте применения технологии VSTA в реальном проекте (см раздел \ref{sec:use_exis_techn}). Кроме того, в рамках данной работы необходимо оценить успешнось разработанной платформы, как замены устаревшего инструментария VBA, так как один из сценариев применения платформы - замена VBA при портировании COM/ActveX-приложений, использующих VBA, на платформу .NET.

\paragraph{Методика сравнительного анализа}

Рассматриваемые плаформы поддержки расширений можно представить как совокупность следующих компонент:

\begin{itemize}
   \item среда разработки расширений
   \item пользовательский интерфайс (инструментарий для работы с расширениями)
   \item ядро, обеспечивающее взаимодействие приложения с расширениями
\end{itemize}

Для получения адекватных результатов сравнительный анализ необходимо проводить по каждой из этих компонент.

\paragraph{среда разработки}

VSTA Studio является логичным развитием VB6 Studio, используемой как редактор VBA-кода. Большинство основных элементов управления сргуппированы аналогично и выполняют аналогичные операции в этих IDE. Однако, VSTA имеет более <<продвинутый>> визуальный редактор форм, а так же новые инструменты IntelliSense, увеличивающие эффективность разработки. Используемая в разработанной платформе IDE SharpDevelop имеет похожий функционал, однако набор возможностей редактора кода не столь широк, как в VSTA Studio. В этом мы убедились ранее, реализовав генератор сигнатур обработчиков событий, отсутствующий в SharpDevelop по умолчанию. (см. раздел \ref{sec:ehsg}) Так же, из-за того что использование этой IDE не подразумевает сценария использования как интегрированой среды разработки расширений, проявилось множество проблем, требующих разрешения. (поднобнее про эти проблемы и методы их решения см. раздел \ref{sec:dev-details}) Не смотря на отсутствие некоторых функций в базовом установочном пакете SharpDevelop, его функционал может быть расширен засчет механизма плагинов, либо интегрированной технологии SDA.

\TODO{мб табличка + критерии}

\paragraph{Выводы}

Средства разработки компании Microsoft предоставляют более шировий набор возможностей <<из коробки>>, однако используемая в разработанной платформе IDE является более гибким и масштабируемым решением. Кроме того, SharpDevelop является полноценной IDE для разработки программ на большом числе .NET-совместимых языков программирования, в то время как продукты Microsoft поддерживают только Visual Basic и имеют менее богатый встроенный инструментарий для разработчика.

\paragraph{Пользовательский интерфейс}

Как таковые, графические средства для проведения операций с расширенями продукты Microsoft не предоставляют. То есть разработчику программного обеспечения необходимо реализовать свой собственный инструментарий в соответствии с поставлеными целями. В разработанную платформу интегрирован пользовательский интерфейс, реализованный на Windows Forms, предоставляющий основные инструменты для управления расширениями. (см. раздел \ref{sec:dev-details}) \TODO{глава про гуи?}

\paragraph{Выводы}

Преимущества такого решения очевидны. Из недостатков стоит отметить неприемлемость такого решения в случае, если втроенные в платформу элементы нарушают целостность визуального оформления приложения. В этом случае имеет смысл использовать один из паттернов проектирования для предоставления разработчику возможность самому реализовать представления компонент графического интерфейса в соответствии со стилем разрабатываемого приложения.

\paragraph{взаимодействие расширения и приложения}

В этом аспекте не имеет смысла сравнивать разработанную платформу с VBA из-за принципиального отличия модели взаимодействия COM-компонент и .NET-библиотек между собой.

\paragraph{выводы}

\TODO{Доделать!}

\pagebreak
