\section{Сравнение с аналогами}

Единственным непосредственным аналогом разработанной платформы является VSTA. (речь идет о ПО для платформы .NET)  Сравнение будет происходить основываясь на опыте применения технологии VSTA в реальном проекте (см раздел \ref{sec:use_exis_techn}). Кроме того, в рамках данной работы необходимо оценить успешнось разработанной платформы, как замены устаревшего инструментария VBA, так как один из сценариев применения платформы - замена VBA при портировании COM/ActveX-приложений, использующих VBA, на платформу .NET.

\subsubsection{Методика сравнительного анализа}

Рассматриваемые плаформы поддержки расширений можно представить как совокупность следующих компонент:

\begin{itemize}
   \item среда разработки расширений;
   \item пользовательский интерфайс (инструментарий для работы с расширениями);
   \item ядро, обеспечивающее взаимодействие приложения с расширениями.
\end{itemize}

Для получения адекватных результатов сравнительный анализ необходимо проводить по каждой из этих компонент. Так как сформулировать критерии для получения точных количественных оценок при сравнении, к примеру, удобства пользовательского интерфейса, не представляется возможным, все выводы будут следовать по большей части личного опыта использования тех или иных продуктов и инструментов, а так же из отзывов других разработчиков или пользователей этих продуктов.

Кроме перечисленных компонент системы немаловажным фактором в выборе того или иного продукта будет являться простота его использования. В конце раздела будет проведено сравнение сложности интеграции продуктов в приложение.

\subsubsection{Среда разработки}

VSTA Studio является логичным развитием VB6 Studio, используемой как редактор VBA-кода. Большинство основных элементов управления сргуппированы аналогично и выполняют аналогичные операции в этих IDE. Однако, VSTA имеет более <<продвинутый>> визуальный редактор форм, а так же новые инструменты IntelliSense, увеличивающие эффективность разработки. Используемая в разработанной платформе IDE SharpDevelop имеет похожий функционал, однако набор возможностей редактора кода не столь широк, как в VSTA Studio. В этом мы убедились ранее, реализовав генератор сигнатур обработчиков событий, отсутствующий в SharpDevelop по умолчанию. (см. раздел \ref{sec:ehsg}) Так же, из-за того что использование этой IDE не подразумевает сценария использования как интегрированой среды разработки расширений, проявилось множество проблем, требующих разрешения. (поднобнее про эти проблемы и методы их решения см. раздел \ref{sec:dev-details}) Не смотря на отсутствие некоторых функций в базовом установочном пакете SharpDevelop, его функционал может быть расширен засчет механизма плагинов, либо интегрированной технологии SDA.

Средства разработки компании Microsoft предоставляют более шировий набор возможностей <<из коробки>>, однако используемая в разработанной платформе IDE является более гибким и масштабируемым решением. Кроме того, SharpDevelop является полноценной IDE для разработки программ на большом числе .NET-совместимых языков программирования (C\#, VisualBasic.NET, F\#, Boo, IronPython, IronRuby и другие, для которых существуют плагины SharpDevelop, поддерживающие их), в то время как продукты Microsoft поддерживают только Visual Basic и имеют менее богатый встроенный инструментарий для разработчика.

\subsubsection{Пользовательский интерфейс}

Как таковые, графические средства для проведения операций с расширенями продукты Microsoft не предоставляют. То есть разработчику программного обеспечения необходимо реализовать свой собственный инструментарий в соответствии с поставлеными целями. В разработанную платформу интегрирован пользовательский интерфейс, реализованный на Windows Forms, предоставляющий основные инструменты для управления расширениями. (см. раздел \ref{sec:macro-gui})

Преимущества такого решения очевидны: разработчику программного обеспечения достаточно встроить готовые компоненты в свое приложение чтобы получить полнофункциональную систему боддержки расширений. Из недостатков стоит отметить неприемлемость такого решения в случае, если втроенные в платформу элементы нарушают целостность визуального оформления приложения. В этом случае имеет смысл использовать один из паттернов проектирования для предоставления разработчику возможность самому реализовать представления компонент графического интерфейса в соответствии со стилем разрабатываемого приложения.

\subsubsection{Взаимодействие расширения и приложения}

В этом аспекте не имеет смысла сравнивать разработанную платформу с VBA из-за принципиального отличия модели взаимодействия COM-компонент и .NET-библиотек между собой. В свою очередь, используя VSTA можно выбрать различные варианты работы с ним. В одном из случаев взаимодействие расширения и приложения будет происходить через System.Addin, особенности этого подхода были подробно описаны в разделе \ref{sec:system_addin}. В разрабатываемой платформе был выбран .NET Reflection в качестве основы для организации взаимодействия. По большому счету это и евляется основным отличием разработанной платформы от VSTA.

Эта часть платформы хоть и играет, пожалуй, самую важную роль, но скрыта от пользовательских глаз. Поэтому, если все пользовательские сценарии реализованы верно и не возникает проблем со стабильностью и производительностью, по большому счету не имеет смысла утверждать что тот или иной подход оказывается лучше или хуже. Минусом исполизуемого подхода (см. раздел \ref{sec:extention_interaction}) является тот факт, что сборки расширения остаются в домене приложения, то есть в памяти, до завершения работы с ним. Это сложно назвать серьезным недостатком, так как эти <<паразитные>> сборки полностью изолируются и не могут повлиять на работу приложения. Более того, их размер ничтожно мал по сравнению с доступными объемами памяти, иварьируется от десятков до сотен килобайт, в зависимости от объема кода расширения. В исключительных случаях, если расширение перегружено разнообразными ресурсами, размер может доходить до нескольких мегабайт. В конце-концов, так как проблема <<паразитных>> сборок имеет место только в случае активного использования отладчика расширения и редактирования его кода, она не будет мешать работе с уже готовыми расширениями.

\subsubsection{Интеграция}

Основным преимуществом разработаного решения является относительная простота его интеграции как в существующее приложение, так и на этапе разработки нового приложения. Простота достигается благодаря использованию разработанного механизма взаимодействия расширения и приложения, в основе которого лежит .NET Reflection. Подробнее про особенности реализованного механизма написано в разделе \ref{sec:extention_interaction}. Для предоставления доступа к объектам программисту требуется всего лишь реализовать этими объектами интерфейс, используемый модулем интеграции расширений разработанной платформы. В то же время, интеграция рассмотренных в обзоре (см. раздел \ref{sec:extention_interaction}) требует реализации сложных и громоздких протоколов взаимодействия на основе контрактов, а так же накладывает некоторые условия на архитектуру разрабатываемого приложения. Конечно, таким образом достигается изоляция расширения от приложения, но при этом интеграция сильно усложнена.

Так же стоит отметить наличие в разработанной платформе средств для управления расширениями, которые могут быть легко добавлены полностью или частично в целевое приложение на этапе интеграции. Если же такой вариант разработчика не устроит, он может реализовать собственные визуальные компоненты, однако, их придется интегрировать в платформу поддержки расширений, на что будет потрачено дополнительное время.

\subsubsection{Выводы}

Исходя из проведенного выше анализа следует, что разработанная платформа имеет ряд преимуществ над большинством существующих решений:

\begin{itemize}
   \item Наличие полноценной среды разработки расширений;
   \item Более простая интерация в существующие приложения;
   \item Возможность отладки расширений;
\end{itemize}

\pagebreak
