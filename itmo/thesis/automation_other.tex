\subsubsection{Другие разработки}

Помимо универсальных платформ для автоматизации, рассмотренных выше, существует множество скриптовых языков, которые могут быть использованы для создания собственной платформы, решающей задачи автоматизации ПО. Некоторые из этих языков (как правило, речь идет о диалектах языка, нацеленных на эффективную работу с одним из программных продуктов) разрабатывались специально для конкретной задачи по автоматизации приложения. Это позволило глубоко интегрировать скриптовый язык в архитектуру приложения, которое он расширяет. Яркими примерами такогих языков являются диалекты языка Lisp. AutoLisp с 1986 года применяется в AutoCAD и неразрывно с ним связан, Emacs Lisp используется в текстовых редакторах GNU Emacs и XEmacs. Особенностью таких узкоспециализированных скриптовых языков является глубокая интеграция с приложением и сложность в освоении. Они хорошо подходят для решения задач, специфичных для конкретного приложения, но их использование как универсального инструмента для автоматизации ПО невозможно.

Так же существует несколько скриптовых языков, совместимых с платформой .NET. Наиболее перспективным и популярным из них является IronPython, на примере которого и будет рассмотрена концепция использования скриптового языка для автоматизации основного приложения.


\subsubsection{IronPython} % TODO :: перенести в отдельный файл?
{\it IronPython} -- реализация языка программирования {\it Python} для платформ {\it .NET Framework} и {\it Mono}. {\it IronPython} полностью написан на {\tt C\#}, и является транслятором компилирующего типа. Код скрипта, написанного на {\it IronPython}, может использоваться одним из следующих способов:
\begin{itemize}
 \item посредством компиляции в независимую сборку, которая в дальнейшем может быть загружена в приложение как зависимось или с помощью {\it .NET Reflection};
 \item посредством размещения {\it IronPython}-подсистемы в основном приложении и динамической трансляции кода.
\end{itemize}

Эта особенность {\it IronPython} очень важна в контексте задач, связанных с автоматизацией ПО.

На схеме ниже рассмотрен типичный способ использования {\it IronPython} с целью автоматизации приложения:

\TODO{ (dennis.yolkin) схема и её описание}.