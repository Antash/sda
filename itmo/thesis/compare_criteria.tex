\TODO{тут много блабла про критерии сравнения и прочий фарш}

\subsubsection{Критерии сравнения}
На основании требований к программному продукту были сформулированы следующие критерии оценки существующих разработок. Критерии приведены в порядке убывания их значимости, хотя удовлетворение всем этим критериям является критичным для реализации замены VBA в рамках проекта по портированию приложения с COM на .NET. Критерии подразумевают различные варианты соответствия. Некоторые – бинарный ответ (удовлетворяет, либо нет), некоторые – предлагают провести сравнительную оценку степени соответствия, некоторые определяют вероятность или прогноз.

\begin{enumerate}
\item Тип продукта (Платформа, Библиотека, Язык, и т.д.);
\item Статус релиза;
\item Возможность реализации расширения конечным пользователем;
\item Наличие IDE для создания кода расширения;
\item Возможность отладки расширения;
\item Сложность интеграции в готовое приложение;
\item Открытость исходного кода;
\item Скрипты или плагины?;
\item Отслеживание зависимостей расширений;
\item Стоимость (и лицензирование в целом).
\end{enumerate}

Следующий этап – создание сравнительной таблицы, которая будет показывать насколько тот или иной продукт удовлетворяет поставленной задаче. Для того, чтобы продукт было возможно использовать, необходимо полное или почти полное удовлетворение всем имеющимся критериям.

\pagebreak
