\begin{landscape}

\begin{table}[H]
  \caption{Сравнительная таблица технологий, подходов и платформ для создания расширяемых и/или автоматизируемых приложений}
  \label{tabular:tech_compare_tab}
  \begin{center}
  \begin{tabular}{|p{4.5cm}|c|c|c|c|c|c|c|c|c|}
  
    \hline
      Критерий &
      MEF &
      Al Platform &
      Plux.NET &
      VBA &
      VSTA/VSTO &
      System.Addin &
      Mono.AddIns &
      IronPython &
      SDA \\
    \hline
      Тип продукта &
      FW &
      FW &
      FW &
      Lang &
      FW &
      Lib &
      Lib &
      Lang &
      Tech \\
    \hline
      Статус релиза &
      A &
      B &
      B &
      S &
      S &
      S &
      S &
      S &
      S \\
    \hline
      Для конечного пользователя? &
      Нет &
      Нет &
      Нет &
      Да &
      Да &
      Нет &
      Нет &
      Да &
      Нет \\
    \hline
      Наличие IDE для создания кода расширения &
      Нет &
      Нет &
      Нет &
      Да &
      Да &
      Нет &
      Нет &
      Нет &
      Нет \\
    \hline
      Возможность отладки расширения &
      Нет &
      Нет &
      Нет &
      Да &
      Да &
      Нет &
      Нет &
      Да &
      Нет \\
    \hline
      Сложность интеграции в готовое приложение &
      ? &
      ? &
      ? &
      + &
      + &
      ? &
      ? &
      + &
      - \\
    \hline
      Открытость исходного кода &
      Нет &
      Нет &
      Нет &
      Нет &
      Нет &
      Нет &
      Да &
      Да &
      Да \\
    \hline
      Скрипты или плагины? &
      Плагин &
      Плагин &
      Плагин &
      Скрипт &
      Оба &
      Плагин &
      Плагин &
      Скрипт &
      Плагин \\
    \hline
      Отслеживание зависимостей расширений &
      Да &
      Да &
      Да &
      Нет &
      Нет &
      Нет &
      Нет &
      Нет &
      Да \\
    \hline
      Лицензирование &
      MPL &
      PR &
      FreeWare &
      EULA &
      EULA &
      EULA &
      MIT &
      GPL &
      GPL \\
    \hline
    
  \end{tabular}
  \end{center}
\end{table}

\end{landscape}

\subsubsection{Условные обозначения}

{\bf Тип продукта}

\begin{itemize}
	\item FW: (Framework) Готовая платформа для внедрения комплекстой поддержки расширений в разрабатываемое приложение.
	\item Lib: (Library) Библеотека, сторонняя, или встроенная в платформу .NET, предоставляющая инструменты реализации процессов и механизмов, необходимых для организации работы с расширениями или скриптами.
	\item Lang: Язык программирования (скриптовый), имеющий возможность простой интеграции в приложение как средство реализации макросов. 
	\item Tech: Технология, позволяющая решать задачу интеграции возможностей расширяемости, но не являющаяся платформой.
\end{itemize}

{\bf Статус релиза}

\begin{itemize}
	\item A: (Alpha) Нестабильная версия для внутреннего тестирования. Может быть использована в реальном проекте с рядом допущений и рисков.
	\item В: (Beta) Нестабильная версия для публичного тестирования. Может быть использована в реальном проекте с рядом допущений и рисков.
	\item S: (Stable release) Стабильная RTM версия, может быть использована без огроничений.
\end{itemize}

{\bf Сложность интеграции в готовое приложение}

\begin{itemize}
	\item +: Интеграция в готовое приложение возможна, но требует значительных усилий. Относится, как правило, к средствам интеграции макросов.
	\item -: Интеграция в готовое приложение невозможна.
	\item ?: Сложность интеграции сложно оценить, так как она сильно зависит от архитектуры приложения и требуемой глубины интеграции. Относится, как правило, к платформам для поддержки расширений.
\end{itemize}

{\bf Лицензирование}

\begin{itemize}
	\item MPL: Mozilla Public License
	\item FreeWare: Бесплатное ПО, распространяемое без исходного кода.
	\item PR: Проприетарная лицензия. При приодретении Al Platform частично предоставляется исходный код.
	\item GPL: General Public License
	\item EULA: Пользовательское соглашение компании Microcoft.
\end{itemize}