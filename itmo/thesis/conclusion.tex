\setcounter{secnumdepth}{0}
\section{Заключение}
\setcounter{secnumdepth}{2}

В данной работе затронута одна из проблем, возникающих при разработке современныз программных комплексов: проблема автоматизации и расширения ПО. Большинство современных крупных приложений так или иначе поддерживают автоматизацию либо расширение. В некоторых ситуациях для решения этой задачи используются какие-нибудь готовые разработки, в некоторых случаях реализуется какой-то специфичный механизм. Существует несколько готовых программных решений для интеграции возможностей автоматизации и расширения в приложения. В работе рассмотрены наиболее популярные из них. 

Необходимость интеграции возможностей автоматизации и расширения в приложение возникла на реальном коммерческом проекте. В рамках проекта необходимо было портировать приложение, разработанное с использованием устаревших технологий, на современные платформы. Автоматизация и расширение достигались в приложении за счёт {\it Visual Basic for Applications}, которая, хоть и является на настоящий момент устаревшей, весьма успешно решала поставленную задачу. Найти подходящую замену для {\tt .NET} оказалось непросто. Наилучшим кандидатом казалась платформа {\it Visual Studio Tools for Applications}, которая и была внедрена в разрабатываемое приложение. Однако как на этапе разработки, так и не этапе тестирования возникло множество проблем. Более того, на момент окончания разработки выяснилось, что {\it VSTA} больше не поддерживается и лицензию на неё приобрести невозможно.

В результате было принято решение разрабатывать новую платформу, позволяющую интегрировать возможности расширения и автоматизации в приложения. При разработке платформы учитывались результаты исследования существующих решений, а также опыт внедрения одного из них на реальном проекте, разрабатываемом в компании First Line Software.

Разработанная платформа отвечает всем поставленным требованиям. Она нацелена на упрощение процесса интеграции возможностей автоматизации и расширения в приложения. Помимо модулей, встраиваемых в приложения, платформа содержит ряд утилит для анализа и генерации кода, что ускоряет процесс интеграции.

\pagebreak