\setcounter{secnumdepth}{0} % за счёт этого "Введение" будет без номера
\section{Введение}
\setcounter{secnumdepth}{2}

Современное программное обеспечение (ПО) зачастую представляет из себя крупные программные комплексы, состоящие из множества различных компонентов. В процессе эксплуатации ПО одни компоненты могут устаревать, в других будут обнаружены критические ошибки и уязвимости. Помимо этого некоторым пользователям может оказаться необходим какой-то функционал, отсутствующий в базовой версии. Другие пользователи могут захотеть сконфигурировать приложение под свои нужды, автоматизировать некоторые его части. Все эти проблемы ведут к изучению вопросов, связанных с автоматизацией и расширением программного обеспечения.

Под {\it автоматизаций} программного продукта понимается возможность формировывать, сохранять и воспроизводить в автоматическом режиме некоторую последовательность пользовательских действий. К примеру, в популярном текстовом редакторе {\it Microsoft Word} для ускорения часто выполняемых операций редактирования или форматирования, а также для объединения нескольких команд или автоматизации обработки сложных последовательных действий в задачах можно записать макрос. В дальнейшем можно запускать этот макрос для повторения ранее записанных действий.

Под {\it расширением} программного приложения понимается возможность добавлять в существующий продукт новые модули, расширяющие возможности программы. Примером может служить популярный файловый менеджер {\it Far}, в который включён текстовый редактор. Однако в базовой конфигурации в редакторе нет подсветки синтаксиса языков программирования, в то время как многие разработчики используют его для написания кода. Недостоющий функционал добавляется за счёт планига (расширения) {\it Far Colorer}. 

Программное приложение, в дизайне которого предусмотрены возможности автоматизации и расширения, предоставет целый набор дополнительных преимуществ и для разработчика основного приложения, и для сторонних разработчиков, и для конечного пользователя:
\begin{itemize}
\item конфигурирование приложения конечным ползователем
\item выпуск обновлённых и исправленных компонентов ПО
\item разработка дополнений сторонними программистами
\item автоматизация часто повторяющейся последовательности действий
\item дополнение недостоющего функционала за счёт написания скриптов
\end{itemize}

Раньше эти преимущества достигались за счёт периодических выпусков новых версий программного обеспечения. Это было не очень удобно, так как новые версию появлялись не так часто, и не было средств для решения проблем, содержащихся в текущей версии продукта. Был компромиссный вариант - выпуск пакетов исправления ({\it Service Pack}) и пакетов новых возможностей ({\it Feature pack}). 

Позже, с массовым распространением Интернета, этот подход был заменён автоматическим обнавлением. Суть осталась та же, только процесс установки пакетов исправлений и дополнений автоматически загружался с веб-сайта разработчика и автоматически устанавливался. 

Со временем, наряду с автоматическим обновлением, разработчики программного обеспечения стали добавлять в свои приложения возможности автоматизации и расширения. Этот шаг добавил программным комплексам большую гибкость, а также дал дополнительные возожности сторонним разработчикам. 

\pagebreak