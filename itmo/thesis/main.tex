\documentclass[12pt,a4paper,oneside]{book}

\usepackage[utf8]{inputenc}
\usepackage{ucs}
\usepackage[english,russian]{babel}
\usepackage{extsizes} % чтобы можно было использовать шрифты больше, чем 12pt
\usepackage{pscyr} % красивые русскоязычные шрифты (мануал по установке тут: http://ishutov.blogspot.com/2011/02/miktex-29-pscyr-04d.html)
\usepackage{xcolor}
%\usepackage{listings}
\usepackage{graphicx}
\usepackage{url}

\usepackage{indentfirst}

% для математических символов
\usepackage{amsmath, amsthm, amssymb}

% для построения диаграмм
%\usepackage{tikz}
%\usetikzlibrary{positioning,arrows,shapes,shadows}

\graphicspath{{pics/}}

% поля (в соответствии с Требованиями Оформления Магистерских Работ)
% по умолчанию левое и верхнее и левое поле = 1дюйм (2.54 см). Это соответствует требованиям.
\oddsidemargin = 0mm
\topmargin = -15mm % вычитается размер колонтитула.
\textwidth = 175mm
\textheight = 247mm

% абзацный отступ
\parindent = 12mm

\sloppy
\makeatletter
\renewcommand{\baselinestretch}{1.5} % межстрочный интервал

% чтобы каждая новая глава начиналась с новой страницы
\let\stdsection\section
\renewcommand\section{\newpage\stdsection}

\pagestyle{plain}

%\newcounter{appendix_number}
%\newcommand{\Appendix}[1]{\addtocounter{appendix_number}{1} \section{Приложение \arabic{appendix_number}. #1}}

% замена в подписи к рисунку разделителя ":" на "." (Рис. 2: Рисунок   станет   Рис. 2. Рисунок.) - - - - - - - - -
\renewcommand{\@makecaption}[2]{%
\vspace{\abovecaptionskip}%
\sbox{\@tempboxa}{#1: #2}
\ifdim \wd\@tempboxa >\hsize
#1: #2\par
\else
\global\@minipagefalse
\hbox to \hsize {\hfil #1. #2\hfil}%
\fi
\vspace{\belowcaptionskip}}
% - - - - - - - - - - - - - - - - - - - - - - - - - - - - - - - - - - - - - - - - - - - - - - - - - - - - - - - - - 


% - - - - - - - - - - - - - - - - - - - - - - - - - - - - - - - - - - - - - - - - - - - - - - - - - - - - - - - - - 
% оформление заголовков глав
 \def\onelineskip{\normalsize\vskip\baselineskip}

 \renewcommand{\@makechapterhead}[1]{%
 {\raggedright
 \parindent=0cm\hangafter=1%
 \newbox\numberbox
 \setbox\numberbox\hbox{\normalfont\LARGE\textar{\bfseries\thechapter.\hskip0.5em}}%
 \hangindent=\wd\numberbox %
 \normalfont\LARGE\textar{\@chapapp{}\bfseries
 \unhbox\numberbox #1}\par
 \nopagebreak
 \onelineskip
 }}

 \renewcommand{\@makeschapterhead}[1]{%
 {\parindent=0pt \raggedright
 \normalfont\LARGE\textar{\bfseries #1}\par
 \nopagebreak
 \onelineskip
 }}
% - - - - - - - - - - - - - - - - - - - - - - - - - - - - - - - - - - - - - - - - - - - - - - - - - - - - - - - - - 






\begin{document}
    %\renewcommand{\@oddhead}{\hfill \thepage \hfill} % номера страниц по центру сверху
    \renewcommand{\thesection}{\arabic{section}}
    \renewcommand{\bibname}{Источники}
    
    \newcommand{\TODO}[1]{\fcolorbox{red}{red}{TODO :: #1}}
    \newcommand{\Definition}[1]{{\bf Определение. } #1 \par} % TODO :: Добавить счётчик
   
    \setcounter{page}{5}   % номер первой страницы
	
    %\tableofcontents
    %\pagebreak
    
    
	% ---------  Основные главы -----------------------------------
	{\itЕбал} я эту магистерскую.
	
	\TODO{ХУЙ!}
	
	The quadratic formula is $$-b \pm \sqrt{b^2 - 4ac} \over 2a$$

	% -------------------------------------------------------------
    
    
    % Библиография.
    % Как описано здесь: http://en.wikibooks.org/wiki/LaTeX/Bibliography_Management
    % TODO :: Установить требуемый Правилами стиль библиографии.
    %\bibliographystyle{unsrt} % в порядке упоминания в тексте
    %\cleardoublepage                           % !!!!! TODO :: такое решение сработает вроде как не всегда, иногда будет пустая страница !!!!!
    %\addcontentsline{toc}{section}{Источники} % !!!!! TODO :: это баг!!!!!!! литература начнётся с новой страницы !!!!!!!!!!!!!!!!!!!!!!!!!!!
    %\bibliography{references}
    
    %\appendix
    
\end{document}
