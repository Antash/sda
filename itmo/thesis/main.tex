\documentclass[12pt,a4paper,oneside]{book}

\usepackage[utf8]{inputenc}
\usepackage{ucs}
\usepackage[english,russian]{babel}
\usepackage{extsizes} % чтобы можно было использовать шрифты больше, чем 12pt
\usepackage{pscyr} % красивые русскоязычные шрифты (мануал по установке тут: http://ishutov.blogspot.com/2011/02/miktex-29-pscyr-04d.html)
\usepackage{xcolor}
%\usepackage{listings}
\usepackage{graphicx}
\usepackage{url}

\usepackage{indentfirst}

% для математических символов
\usepackage{amsmath, amsthm, amssymb}

% для построения диаграмм
%\usepackage{tikz}
%\usetikzlibrary{positioning,arrows,shapes,shadows}

\graphicspath{{pics/}}

% поля (в соответствии с Требованиями Оформления Магистерских Работ)
% по умолчанию левое и верхнее и левое поле = 1дюйм (2.54 см). Это соответствует требованиям.
\oddsidemargin = 0mm
\topmargin = -15mm % вычитается размер колонтитула.
\textwidth = 175mm
\textheight = 247mm

% абзацный отступ
\parindent = 12mm

\sloppy
\makeatletter
\renewcommand{\baselinestretch}{1.5} % межстрочный интервал

% чтобы каждая новая глава начиналась с новой страницы
\let\stdsection\section
\renewcommand\section{\newpage\stdsection}

\pagestyle{plain}

%\newcounter{appendix_number}
%\newcommand{\Appendix}[1]{\addtocounter{appendix_number}{1} \section{Приложение \arabic{appendix_number}. #1}}

% замена в подписи к рисунку разделителя ":" на "." (Рис. 2: Рисунок   станет   Рис. 2. Рисунок.) - - - - - - - - -
\renewcommand{\@makecaption}[2]{%
\vspace{\abovecaptionskip}%
\sbox{\@tempboxa}{#1: #2}
\ifdim \wd\@tempboxa >\hsize
#1: #2\par
\else
\global\@minipagefalse
\hbox to \hsize {\hfil #1. #2\hfil}%
\fi
\vspace{\belowcaptionskip}}
% - - - - - - - - - - - - - - - - - - - - - - - - - - - - - - - - - - - - - - - - - - - - - - - - - - - - - - - - - 


% - - - - - - - - - - - - - - - - - - - - - - - - - - - - - - - - - - - - - - - - - - - - - - - - - - - - - - - - - 
% оформление заголовков глав
 \def\onelineskip{\normalsize\vskip\baselineskip}

 \renewcommand{\@makechapterhead}[1]{%
 {\raggedright
 \parindent=0cm\hangafter=1%
 \newbox\numberbox
 \setbox\numberbox\hbox{\normalfont\LARGE\textar{\bfseries\thechapter.\hskip0.5em}}%
 \hangindent=\wd\numberbox %
 \normalfont\LARGE\textar{\@chapapp{}\bfseries
 \unhbox\numberbox #1}\par
 \nopagebreak
 \onelineskip
 }}

 \renewcommand{\@makeschapterhead}[1]{%
 {\parindent=0pt \raggedright
 \normalfont\LARGE\textar{\bfseries #1}\par
 \nopagebreak
 \onelineskip
 }}
% - - - - - - - - - - - - - - - - - - - - - - - - - - - - - - - - - - - - - - - - - - - - - - - - - - - - - - - - - 






\begin{document}
    %\renewcommand{\@oddhead}{\hfill \thepage \hfill} % номера страниц по центру сверху
    \renewcommand{\thesection}{\arabic{section}}
    \renewcommand{\bibname}{Источники}
    
    \newcommand{\TODO}[1]{\fcolorbox{red}{red}{TODO :: #1}}
    \newcommand{\Definition}[1]{{\bf Определение. } #1 \par} % TODO :: Добавить счётчик
   
    \setcounter{page}{5}   % номер первой страницы
	
    %\tableofcontents
    %\pagebreak
    
    % Аннотация
	\setcounter{secnumdepth}{0}
\section*{Аннотация}
\setcounter{secnumdepth}{2}

При разработке современных программных комплексов зачастую возникают задачи, связанные с поддержкой функций расширения и автоматизации программного обеспечения. Это необходимо для реализации возможности конфигурирования, настройки, переопределения поведения, а также реализации недостоющего функционала приложения конечным пользователем. Чаще всего реализация данных возможностей достигается за счёт функций автоматического обновления программного обеспечения, поддержки плагинов сторонних производителей, наличия SDK (Software Development Kit) для разработки расширений, поддержки скриптов. С точки зрения числа предоставляемых возможностей и гибкости наибольший интерес представляет последний вариант. В зависимости от специфики разрабатываемого ПО могут быть различные сценарии использования скриптов: простейшее конфигурирование приложения конечным пользователем, разработка дополнений, расширяющих возможности приложения, распространение скриптов, разработанных сторонними производителями. В настоящее время многие разработчики предоставляют возможность использовать скрипты в своих приложениях. Помимо этого, существуют специальные технологии и платформы для интеграции скриптовых движков в приложения: VBA (Visual Basic for Applications), VSTA (Visual Studio Tools for Applications), VSTO (Visual Studio Tools for Office), IronPython, Game Maker Language) и другие. Основная проблема заклюается в том, чаще всего в каждом семействе программных продуктов разрабатываются собственные инструменты для интеграции поддержки скриптов в ПО. Существующие <<универсальные>> средства для решения этой задачи оказываются неэффективными и неудобными для использования на реальных проектах.

Целью данной работы является исследование существующих решений в области расширения и автоматизации ПО, а также создание универсальной платформы, включающей в себя инструменты для внедрения возможностей автоматизации и расширения в разрабатываемые программные продукты. Помимо самого механизма скриптов, разрабатываются утилиты для автоматического анализа кода основной программы и генерации вспомогательного кода. Платформа включает в себя механизм для отладки скриптов. Для достижения наибольшей эффектисности и охвата максимального числа сценариев использования в предложенном подходе объединяются возможности использования как интерпретируемых, так и компилируемых языков. 

На примере коммерческого проекта и проекта с открытым исходным кодом будут сравниваться возможности существующих продуктов и разработанной платформы. 
\pagebreak
	
	% ---------  Основные главы -----------------------------------
	% -------------------------------------------------------------
    
    
    % Библиография.
    % Как описано здесь: http://en.wikibooks.org/wiki/LaTeX/Bibliography_Management
    % TODO :: Установить требуемый Правилами стиль библиографии.
    %\bibliographystyle{unsrt} % в порядке упоминания в тексте
    %\cleardoublepage                           % !!!!! TODO :: такое решение сработает вроде как не всегда, иногда будет пустая страница !!!!!
    %\addcontentsline{toc}{section}{Источники} % !!!!! TODO :: это баг!!!!!!! литература начнётся с новой страницы !!!!!!!!!!!!!!!!!!!!!!!!!!!
    %\bibliography{references}
    
    %\appendix
    
\end{document}
