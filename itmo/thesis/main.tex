\documentclass[12pt,a4paper,oneside]{book}

\usepackage[utf8]{inputenc}
\usepackage{ucs}
\usepackage[english,russian]{babel}
\usepackage{extsizes} % чтобы можно было использовать шрифты больше, чем 12pt
\usepackage{pscyr} % красивые русскоязычные шрифты (мануал по установке тут: http://ishutov.blogspot.com/2011/02/miktex-29-pscyr-04d.html)
\usepackage{xcolor}
%\usepackage{listings}
\usepackage{graphicx}
\usepackage{url}

\usepackage{indentfirst}

% для математических символов
\usepackage{amsmath, amsthm, amssymb}

% для построения диаграмм
%\usepackage{tikz}
%\usetikzlibrary{positioning,arrows,shapes,shadows}

\graphicspath{{pics/}}

% поля (в соответствии с Требованиями Оформления Магистерских Работ)
% по умолчанию левое и верхнее и левое поле = 1дюйм (2.54 см). Это соответствует требованиям.
\oddsidemargin = 0mm
\topmargin = -15mm % вычитается размер колонтитула.
\textwidth = 175mm
\textheight = 247mm

% абзацный отступ
\parindent = 12mm

\sloppy
\makeatletter
\renewcommand{\baselinestretch}{1.5} % межстрочный интервал

% чтобы каждая новая глава начиналась с новой страницы
\let\stdsection\section
\renewcommand\section{\newpage\stdsection}

\pagestyle{plain}

%\newcounter{appendix_number}
%\newcommand{\Appendix}[1]{\addtocounter{appendix_number}{1} \section{Приложение \arabic{appendix_number}. #1}}

% замена в подписи к рисунку разделителя ":" на "." (Рис. 2: Рисунок   станет   Рис. 2. Рисунок.) - - - - - - - - -
\renewcommand{\@makecaption}[2]{%
\vspace{\abovecaptionskip}%
\sbox{\@tempboxa}{#1: #2}
\ifdim \wd\@tempboxa >\hsize
#1: #2\par
\else
\global\@minipagefalse
\hbox to \hsize {\hfil #1. #2\hfil}%
\fi
\vspace{\belowcaptionskip}}
% - - - - - - - - - - - - - - - - - - - - - - - - - - - - - - - - - - - - - - - - - - - - - - - - - - - - - - - - - 


% - - - - - - - - - - - - - - - - - - - - - - - - - - - - - - - - - - - - - - - - - - - - - - - - - - - - - - - - - 
% оформление заголовков глав
 \def\onelineskip{\normalsize\vskip\baselineskip}

 \renewcommand{\@makechapterhead}[1]{%
 {\raggedright
 \parindent=0cm\hangafter=1%
 \newbox\numberbox
 \setbox\numberbox\hbox{\normalfont\LARGE\textar{\bfseries\thechapter.\hskip0.5em}}%
 \hangindent=\wd\numberbox %
 \normalfont\LARGE\textar{\@chapapp{}\bfseries
 \unhbox\numberbox #1}\par
 \nopagebreak
 \onelineskip
 }}

 \renewcommand{\@makeschapterhead}[1]{%
 {\parindent=0pt \raggedright
 \normalfont\LARGE\textar{\bfseries #1}\par
 \nopagebreak
 \onelineskip
 }}
% - - - - - - - - - - - - - - - - - - - - - - - - - - - - - - - - - - - - - - - - - - - - - - - - - - - - - - - - - 






\begin{document}
    %\renewcommand{\@oddhead}{\hfill \thepage \hfill} % номера страниц по центру сверху
    \renewcommand{\thesection}{\arabic{section}}
    \renewcommand{\bibname}{Источники}
    
    \newcommand{\TODO}[1]{\fcolorbox{red}{red}{TODO :: #1}}
    \newcommand{\Definition}[1]{{\bf Определение. } #1 \par} % TODO :: Добавить счётчик
   
    \setcounter{page}{4}   % номер первой страницы
	
    
    % Аннотация
	\setcounter{secnumdepth}{0}
\section*{Аннотация}
\setcounter{secnumdepth}{2}

При разработке современных программных комплексов зачастую возникают задачи, связанные с поддержкой функций расширения и автоматизации программного обеспечения. Это необходимо для реализации возможности конфигурирования, настройки, переопределения поведения, а также реализации недостоющего функционала приложения конечным пользователем. Чаще всего реализация данных возможностей достигается за счёт функций автоматического обновления программного обеспечения, поддержки плагинов сторонних производителей, наличия SDK (Software Development Kit) для разработки расширений, поддержки скриптов. С точки зрения числа предоставляемых возможностей и гибкости наибольший интерес представляет последний вариант. В зависимости от специфики разрабатываемого ПО могут быть различные сценарии использования скриптов: простейшее конфигурирование приложения конечным пользователем, разработка дополнений, расширяющих возможности приложения, распространение скриптов, разработанных сторонними производителями. В настоящее время многие разработчики предоставляют возможность использовать скрипты в своих приложениях. Помимо этого, существуют специальные технологии и платформы для интеграции скриптовых движков в приложения: VBA (Visual Basic for Applications), VSTA (Visual Studio Tools for Applications), VSTO (Visual Studio Tools for Office), IronPython, Game Maker Language) и другие. Основная проблема заклюается в том, чаще всего в каждом семействе программных продуктов разрабатываются собственные инструменты для интеграции поддержки скриптов в ПО. Существующие <<универсальные>> средства для решения этой задачи оказываются неэффективными и неудобными для использования на реальных проектах.

Целью данной работы является исследование существующих решений в области расширения и автоматизации ПО, а также создание универсальной платформы, включающей в себя инструменты для внедрения возможностей автоматизации и расширения в разрабатываемые программные продукты. Помимо самого механизма скриптов, разрабатываются утилиты для автоматического анализа кода основной программы и генерации вспомогательного кода. Платформа включает в себя механизм для отладки скриптов. Для достижения наибольшей эффектисности и охвата максимального числа сценариев использования в предложенном подходе объединяются возможности использования как интерпретируемых, так и компилируемых языков. 

На примере коммерческого проекта и проекта с открытым исходным кодом будут сравниваться возможности существующих продуктов и разработанной платформы. 
\pagebreak
	
	\tableofcontents
    \pagebreak
	
	% ---------  Основные главы -----------------------------------
	\setcounter{secnumdepth}{0} % за счёт этого "Введение" будет без номера
\section{Введение}
\setcounter{secnumdepth}{2}

Современное программное обеспечение (ПО) зачастую представляет из себя крупные программные комплексы, состоящие из множества различных компонентов. В процессе эксплуатации ПО одни компоненты могут устаревать, в других будут обнаружены критические ошибки и уязвимости. Помимо этого некоторым пользователям может оказаться необходим какой-то функционал, отсутствующий в базовой версии. Другие пользователи могут захотеть сконфигурировать приложение под свои нужды, автоматизировать некоторые его части. Все эти проблемы ведут к изучению вопросов, связанных с автоматизацией и расширением программного обеспечения.

Под {\it автоматизацией} программного продукта понимается возможность формировывать, сохранять и воспроизводить в автоматическом режиме некоторую последовательность пользовательских действий. К примеру, в популярном текстовом редакторе {\it Microsoft Word} для ускорения часто выполняемых операций редактирования или форматирования, а также для объединения нескольких команд или автоматизации обработки сложных последовательных действий в задачах можно записать макрос. В дальнейшем можно запускать этот макрос для повторения ранее записанных действий.

Под {\it расширением} программного приложения понимается возможность добавлять в существующий продукт новые модули, расширяющие возможности программы. Примером может служить популярный файловый менеджер {\it Far}, в который включён текстовый редактор. Однако в базовой конфигурации в редакторе нет подсветки синтаксиса языков программирования, в то время как многие разработчики используют его для написания кода. Недостоющий функционал добавляется за счёт планига (расширения) {\it Far Colorer}. 

Программное приложение, в дизайне которого предусмотрены возможности автоматизации и расширения, предоставет целый набор дополнительных преимуществ и для разработчика основного приложения, и для сторонних разработчиков, и для конечного пользователя:
\begin{itemize}
\item конфигурирование приложения конечным ползователем;
\item выпуск обновлённых и исправленных компонентов ПО;
\item разработка дополнений сторонними программистами;
\item автоматизация часто повторяющейся последовательности действий;
\item дополнение недостоющего функционала за счёт написания скриптов.
\end{itemize}

Раньше эти преимущества достигались за счёт периодических выпусков новых версий программного обеспечения. Это было не очень удобно, так как новые версии появлялись не так часто, и не было средств для решения проблем, содержащихся в текущей версии продукта. Был компромиссный вариант --- выпуск пакетов исправления ({\it Service Pack}) и пакетов новых возможностей ({\it Feature Pack}). 

Позже, с массовым распространением Интернета, этот подход был заменён автоматическим обнавлением. Суть осталась та же, только  пакеты исправлений и дополнений автоматически загружались с веб-сайта разработчика и автоматически устанавливались. 

Со временем, наряду с автоматическим обновлением, разработчики программного обеспечения стали добавлять в свои приложения возможности автоматизации и расширения. Этот шаг добавил программным комплексам большую гибкость, а также предоставил дополнительные возожности сторонним разработчикам. 

\pagebreak
	\section{Теоретическое исследование}

\subsection{Расширяемость и автоматизация как единое понятие}
\label{sec:autom-and-ext-as-one-thing}

Во введении были даны определения расширяемости и автоматизации программного обеспечения. С одной стороны, эти два понятия имеют мало общего и могут существовать раздельно. Действительно, программный продукт может предоставлять возможность использовать скрипты для автоматизации, но не поддерживать плагины-расширения. Или наоборот, для приложения может быть разработана развитая система плагинов с автоматическим контролем и обновлением версий, с удобным <<магазином>> расширений, но поддержка автоматизации может отсутствовать.

Как будет показано далее, некоторые существующие решения для интеграции возможностей автоматизации и расширяемости поддерживают сразу оба механизма, некоторые --- только что-то одно. Этому есть объяснение. В некоторых случаях нет необходимости реализовывать оба подхода. Этот вывод подтверждается практикой --- многие современные приложения поддерживают либо автоматизацию, либо расширения, и этого бывает достаточно. Но, как говорилось во введении, данные вопросы встают более остро в крупных программных комплексах, состоящих из множества компонентов. В таких ситуациях зачастую актуальны оба понятия. В связи с этим был сделан вывод о том, что понятия автоматизации и расширяемости можно рассматривать как одно целое. Более того, если это учесть при дизайне приложения, реализовывать поддержку этих механизмов также можно совместно и сделать их неразделимыми.




Для написания пользовательских расширений могут использоваться как скрипты (в терминологии некоторых программ «макросы»)~\cite{automata-via-macros}, так и плагины (независимые модули, написанные на компилируемых языках)~\cite{addins-and-extensibility}.

\subsubsection{Скриптовый язык удобен в следующих случаях:}

\begin{enumerate}
\item Если нужно обеспечить программируемость без риска дестабилизировать систему. Так как, в отличие от плагинов, скрипты интерпретируются, а не компилируются, неправильно написанный скрипт выведет диагностическое сообщение, а не приведёт к системному краху;
\item Если важен выразительный код. Во-первых, чем сложнее система, тем больше кода приходится писать «потому, что это нужно. Во-вторых, в скриптовом языке может быть совсем другая концепция программирования, чем в основной программе — например, игра может быть монолитным однопоточным приложением, в то время как управляющие персонажами скрипты выполняются параллельно. В-третьих, скриптовый язык имеет собственный проблемно-ориентированный набор команд, и одна строка скрипта может делать то же, что несколько десятков строк на традиционном языке. Как следствие, на скриптовом языке может писать программист очень низкой квалификации — например, дизайнер своими руками, не полагаясь на программистов, может корректировать правила игры;
\item Если требуется кроссплатформенность. Хорошим примером является JavaScript — его исполняют браузеры под самыми разными ОС.
\end{enumerate}

\subsubsection{У плагинов же есть три важных преимущества:}

\begin{enumerate}
\item Готовые программы, оттранслированные в машинный код, выполняются значительно быстрее скриптов, которые интерпретируются из исходного кода динамически при каждом исполнении. Поэтому скриптовые языки не применяются для написания программ, требующих оптимальности и быстроты исполнения. Но из-за простоты они часто применяются для написания небольших, одноразовых («проблемных») программ;
\item Полный доступ к любому аппаратному обеспечению или ресурсу ОС (в скриптовом языке для этого должен существовать написанный на машинном коде API). Плагины, работающие с аппаратным обеспечением, традиционно называют драйверами;
\item Если предполагается интенсивный обмен данными между основной программой и пользовательским расширением, для плагина его обеспечить проще.
\end{enumerate}

Также в плане быстродействия скриптовые языки можно разделить на языки динамического разбора (sh) и предварительно компилируемые (perl). Языки динамического разбора считывают инструкции из файла программы минимально требующимися блоками, и исполняют эти блоки, не читая дальнейший код. Предкомпилируемые языки транслируют всю программу в байт-код и затем исполняют его. Некоторые скриптовые языки имеют возможность компиляции программы «на лету» в машинный код (т. н. JIT-компиляция).

\subsection{Формулировка задачи}

В рамках портирования с COM на .NET крупного приложения для управления инвестиционными портфелями потребовалось замена технологии VBA на аналогичною .NET совместимую технологию. Таким образом, потребовалось провести исследование рынка ПО и поиск продуктов с аналогичным или похожим функционалом.

\subsubsection{Основные требования к программному продукту}

\begin{enumerate}
\item Возможность создания расширений для конечных пользователей;
\item Простота использования (Не требует SDK и другого ПО для создания расширений. Вся работа происходит в среде разработки (IDE));
\item Возможность работы как под x86, так и под x64 архитектурами;
\item Возможность отладки расширения;
\item Доступ расширения к объектам расширяемого приложения, реакция на его события;
\item При изменении удалении или добавлении объекта в хост приложении, расширение должно автоматически обновлять информацию о своем окружении;
\item Доступ к библиотеке классов .NET Framework;
\item Поддержка большого числа расширений и взаимодействие их друг с другом;
\item Возможность разрешения зависимостей между расширениями;
\item Удобные инструменты для написания кода (Аналогично инструментам IntelliSense в Microsoft Visual Studio);
\item Возможность сохранения расширений по усмотрению пользователя в базу данных, архив, папку и т. д.;
\item Динамическая загрузка и выгрузка расширений из адресного пространства хост приложения;
\item Компиляция и перезагрузка расширения «на лету» (Не требуется перезапуск хост приложения).
\end{enumerate}

\subsection{Применимость существующих решений}
\paragraph{VBA}
\paragraph{VSTA}
\paragraph{VSTO}


\subsection{Разработка новой платформы для решения поставленной задачи}

\pagebreak
	% -------------------------------------------------------------
    
    
    % Библиография.
    % Как описано здесь: http://en.wikibooks.org/wiki/LaTeX/Bibliography_Management
    % TODO :: Установить требуемый Правилами стиль библиографии.
    %\bibliographystyle{unsrt} % в порядке упоминания в тексте
    %\cleardoublepage                           % !!!!! TODO :: такое решение сработает вроде как не всегда, иногда будет пустая страница !!!!!
    %\addcontentsline{toc}{section}{Источники} % !!!!! TODO :: это баг!!!!!!! литература начнётся с новой страницы !!!!!!!!!!!!!!!!!!!!!!!!!!!
    %\bibliography{references}
    
    %\appendix
    
\end{document}
