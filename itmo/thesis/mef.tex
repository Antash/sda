\subsubsection{MEF (Microsoft Extensibility Framework)}

The Managed Extensibility Framework or MEF is a library for creating lightweight, extensible applications. It allows application developers to discover and use extensions with no configuration required. It also lets extension developers easily encapsulate code and avoid fragile hard dependencies. MEF not only allows extensions to be reused within applications, but across applications as well.

The Problem of Extensibility
Imagine that you are the architect of a large application that must provide support for extensibility. Your application has to include a potentially large number of smaller components, and is responsible for creating and running them.

The simplest approach to the problem is to include the components as source code in your application, and call them directly from your code. This has a number of obvious drawbacks. Most importantly, you cannot add new components without modifying the source code, a restriction that might be acceptable in, for example, a Web application, but is unworkable in a client application. Equally problematic, you may not have access to the source code for the components, because they might be developed by third parties, and for the same reason you cannot allow them to access yours.

A slightly more sophisticated approach would be to provide an extension point or interface, to permit decoupling between the application and its components. Under this model, you might provide an interface that a component can implement, and an API to enable it to interact with your application. This solves the problem of requiring source code access, but it still has its own difficulties.

Because the application lacks any capacity for discovering components on its own, it must still be explicitly told which components are available and should be loaded. This is typically accomplished by explicitly registering the available components in a configuration file. This means that assuring that the components are correct becomes a maintenance issue, particularly if it is the end user and not the developer who is expected to do the updating.

In addition, components are incapable of communicating with one another, except through the rigidly defined channels of the application itself. If the application architect has not anticipated the need for a particular communication, it is usually impossible.

Finally, the component developers must accept a hard dependency on what assembly contains the interface they implement. This makes it difficult for a component to be used in more than one application, and can also create problems when you create a test framework for components.

What MEF Provides
Instead of this explicit registration of available components, MEF provides a way to discover them implicitly, via composition. A MEF component, called a part, declaratively specifies both its dependencies (known as imports) and what capabilities (known as exports) it makes available. When a part is created, the MEF composition engine satisfies its imports with what is available from other parts.

This approach solves the problems discussed in the previous section. Because MEF parts declaratively specify their capabilities, they are discoverable at runtime, which means an application can make use of parts without either hard-coded references or fragile configuration files. MEF allows applications to discover and examine parts by their metadata, without instantiating them or even loading their assemblies. As a result, there is no need to carefully specify when and how extensions should be loaded.

In addition to its provided exports, a part can specify its imports, which will be filled by other parts. This makes communication among parts not only possible, but easy, and allows for good factoring of code. For example, services common to many components can be factored into a separate part and easily modified or replaced.

Because the MEF model requires no hard dependency on a particular application assembly, it allows extensions to be reused from application to application. This also makes it easy to develop a test harness, independent of the application, to test extension components.

An extensible application written by using MEF declares an import that can be filled by extension components, and may also declare exports in order to expose application services to extensions. Each extension component declares an export, and may also declare imports. In this way, extension components themselves are automatically extensible.