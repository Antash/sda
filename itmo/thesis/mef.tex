\subsubsection{MEF (Managed Extensibility Framework)}

{\it MEF} -- библиотека для создания расширяемых приложений~\cite{mef-website}. Она позволяет разработчикам программного обеспечения использовать расширения без необходимости дополнительного конфигурирования. Благодаря {\it MEF}, исходный код расширения полностью независим и не возникает проблем с сильной связностью компонентов приложения и расширения. С помощью {\it MEF} можно разрабатывать как расширения для конкретного приложения, так и универсальные расширения, которые могут быть интегрированы в самые различные продукты.

Обычно в основе любой системы плагинов лежит некий общий интерфейс, который должен быть реализован любым расширением. Интерфейс описывает способы взаимодействия основного приложения и расширения. Это накладывает дополнительное ограничение, добавляя лишнюю зависимость между приложением и плагином. Помимо этого, никак не регламентируется способ взаимодействия между различными плагинами. Ситуация, в которой один плагин зависит от набора других плагинов (и может функционировать только если разрешены все зависимости), в общем случае оказывается неразрешимой.

Подход, выбранный разработчиками {\it MEF} направлен в первую очередь на упрощение разрешения зависимостей между основным приложением и расширением и между различными расширениями.

Вместо необходимости явной регистрации доступных компонентов-расширений, {\it MEF} предоставляет возможность автоматически неявно обнаруживать подходящие расширения. Любое расширение в {\it MEF} декларативно описывает все свои зависимости (в терминах платформы - {\tt imports}) и предоставляемые возможности ({\tt exports}).

Такое решение решает проблемы с зависимостями, описанные ранее. Раз зависимости и предоставляемый функционал описывается декларативно, любое расширение может быть <<подключено>> к основному приложению в процессе выполнения (без перезапуска приложения и тем более без перекомпиляции исходного кода). Таким образом, нет жёстких зависимостей между расширением и расширяемым приложением, а также между плагинами. Нет необходимости описывать какие-либо параметры в конфигурационных файлах. Вся необходимая информация может быть получена из метаданных, доступных при загрузке сборки с расширением.

За счёт объявления списка зависимостей, ими легко управлять. Каждое расширение <<знает>>, какие расширения необходимы для его корректной работы. За счёт этого достигается простота взаимодействия между расширениями.

Раз {\it MEF-}расширение не требует наличия жёстких зависимостей от конкретного расширяемого приложения, автоматически достигается возможность использовать одно и то же расширение в различных приложениях без необходимости специального конфигурирования. Помимо этого, появляются дополнительные возможности для тестирования: для тестирования или исследования возможностей конкретного расширения не обязательно привязываться к какому-либо существующему программному продукту, достаточно разработать тестовое окружение, к которому можно будет подключить это расширение.
