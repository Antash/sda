Ранее (см. раздел~\ref{sec:requirements}) уже были сформулированы основные требования к разрабатываемой платформе. При проектировании архитектуры платформы должны учитываться эти требования, преимущества и недостатки рассмотренных существующих решений, а также опыт внедрения одного из таких решений на реальном проекте (см. раздел~\ref{sec:use-itm-vsta}).

Вполне логично разделить систему на два компонента:
\begin{itemize}
 \item интерактивная среда разработки, содержащая в себе текстовый редактор, средство для управления проектами, отладчик и прочие дополнительные инструменты, упрщающие процесс разработки скриптов. Этому вопросу посвящён раздел~\ref{sec:ide-integration};
 \item модуль, встраиваемый в основное (расширяемое) приложение.
\end{itemize}

Такое глобальное разделение системы на две части даёт следующие преимущества:
\begin{enumerate}
 \item независимость компонентов. Если возникнет необходимость заменить используемую среду разработки на другую, этого можно будет достичь, минимально меняя второй компонент системы;
 \item стабильность. В случае аварийного завершения (или каких-либо ошибок) внутри среды разработки работа основного приложения с большей долей вероятности не будет затронута;
 \item безопасность. Доступ к данным основного приложения строго определяется протоколом взаимодействия компонентов системы.
\end{enumerate}

Описанные компоненты системы выполняются в разных процессах. Это добавляет им независимости. Помимо этого, забегая вперёд, отметим, что это позволит проще реализовать отладку. Таким образом, нужно выбрать способ межпроцессного взаимодействия, который позволит максимально просто и эффективно взаимодействовать компонентам системы. Этому вопросу посвящен раздел~\ref{sec:ipc}.