\subsection{Интеграция SharpDevelop}

\subsubsection{Кастомизация SharpDevelop}
\label{sec:sd_custom}

Среда разработки SharpDevelop имеет встроенный механизм поддержки плагинов. Более того, ядро IDE представляет из себя всего лишь платформу для поддержки плагинов, частью которой и является SDA, рассмотренная ранее. Функционал SharpDevelop реализуется при помощи целой иерархии взаимосвязанных плагинов, хранящихся в папке {\it \\AddIns}. Каждый плагин имеет свой собственный xml-файл конфигурации {\it *.addin}, в котором размещается информация, необходимая ядру для интеграции плагина. 

Помимо отдельных конфигурационных файлов для кажного плагина, существует отдельно лежащий конфигурационный файл {\it ICSharpCode.SharpDevelop.addin}. Этот файл определяет поведение сборки {\it ICSharpCode.SharpDevelop.dll}, которая сама по себе так же является плагином, но очень крупным. Можно сказать, что эта сборка и реализует базовый фанкционал IDE.

В файле содержится несколько секций, рассмотрим подробно те, которые будут использоваться для кастомизации: 

\begin{itemize}
 \item <AddIn> Заголовочная секция, содержит имя сборки, описание, автора, а так же определяет видимость в менеджере плагинов(видимый через менеджер плагин можно отключить)
 \item <Manifest> Хранит имя и версию сборки
 \item <Runtime> <Import> Определяет сборки, загружаемые ядром.
 \item <Path> Содержит путь к визуальному элементу управления типа <<контейнер>>. Внутри этой секции располагается описание различных элементов, которые находятся в контейнере, а так же их поведение.
   \subitem <MenuItem>, <ToolbarItem> Элементы управления. Могут содержать информацию о связаных ресурсах (иконка, текст, всплывающая подсказка), а так же имя класса-обработчика.
   \subitem <Condition>, <ComplexCondition> Условие, либо составное условие, описывающее будет ли отображаться тот или иной элемент управления в определенных ситуациях.
\end{itemize}

Таким образом при помощи конфигурационных файлов можно реализовать несколько сценариев кастомизации:

\begin{itemize}
 \item Загрузка дополнительных сборок в адресное пространство SharpDevelop
  \subitem с целью подменять обработчики событий существующих элементов управления на свои собственные;
  \subitem с целью создания своих собственных элементов управления различного типа (кнопки, панели инструментов, пункты меню, и т. д.)
 \item Комментирование кода в конфигурационном файле для удаления ненужных элементов управления или выключения загрузки сборок.
 \item Добавление или изменение условий.
\end{itemize}
 
Помимо перечисленных сценариев кастомизации SharpDevelop и его компонент с использованием конфигурационных файлов, есть возможность реализации плагинов для этой среды разработки. В рамках этой работы создание плагинов рассматриваться не будет, так как все необходимые действия удалось совершить при помощи конфигурационного файла.
 
Для решения задачи кастомизации была созданна библиотека, содержащая обработчики следующих событий SharpDevelop (в скобках указано новое действие):

\begin{itemize}
 \item Создание нового проекта (вызывает соответствующий диалог создания расширения);
 \item Открытие существующего проекта (диалог открытия проекта расширения из файла/базы данных/архива);
 \item Сохранение проекта/файла (диалог сохранения проекта в файл/базу данных/архив);
 \item Сборка проекта (в случае успешной сборки, обработчик готовит исполняемый файл расширения к загрузке в адресное пространство приложения);
 \item Старт отладки (загружает файл расширения в адресное пространство приложения и присоединяет к его процессу встроенный в SharpDevelop отладчик);
\end{itemize}

Эти события требуют интеграции в хост-процесс IDE для обеспечения взаимодействия с расшеряемым приложением. Реализация своих собственных обработчиков требовала детального изучения исходного кода и архитектуры SharpDevelop, для обеспечения стабильности такого решения. Пример исходного кода обработчика с комментариями можно найти в приложении ?; фрагменты измененного файла конфигурации {\it ICSharpCode.SharpDevelop.addin} см. приложение ?.

\TODO{Как в техе хорошо сверстать код?} 
\TODO{Добавить ссылки на файлы приложений}
 
\subsubsection{Управление сборками расширений}
\label{sec:sd_dll}


\subsubsection{Интеграция с отладчиком}
\label{sec:sd_debug}


\subsubsection{Реализация генератора сигнатур обработчиков событий}
\label{sec:sd_ehsg}


\pagebreak
