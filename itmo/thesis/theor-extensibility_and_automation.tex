\subsection{Расширяемость и автоматизация как единое понятие}

Во введении были даны определения расширяемости и автоматизации программного обеспечения. С одной стороны, эти два понятия имеют мало общего и могут существовать раздельно. Действительно, программный продукт может предоставлять возможность использовать скрипты для автоматизации, но не поддерживать плагины-расширения. Или наоборот, для приложения может быть разработана развитая система плагинов с автоматическим контролем и обновлением версий, с удобным <<магазином>> расширений, но поддержка автоматизации может отсутствовать.

Как будет показано далее, некоторые существующие решения для интеграции возможностей автоматизации и расширяемости поддерживают сразу оба механизма, некоторые - только что-то одно. Этому есть объяснение. В некоторых случаях нет необходимости реализовывать оба подхода. Этот вывод подтверждается практикой - многие современные приложениея поддерживают либо автоматизацию, либо расширения, и этого бывает достаточно. Но, как говорилось во введении, данные вопросы встают более остро в крупных программных комплексах, состоящих из множества компонентов. В таких ситуациях зачастую актуальны оба понятия. В связи с этим был сделан вывод о том, что понятия автоматизации и расширяемости можно рассматривать как одно целое. Более того, если это учесть при дизайне приложения, реализовывать поддержку этих механизмов также можно совместно и сделать их нераздельными.

