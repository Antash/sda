\section{Теоретическое исследование}

\subsection{Расширяемость и автоматизация как единое понятие}
\label{sec:autom-and-ext-as-one-thing}

Во введении были даны определения расширяемости и автоматизации программного обеспечения. С одной стороны, эти два понятия имеют мало общего и могут существовать раздельно. Действительно, программный продукт может предоставлять возможность использовать скрипты для автоматизации, но не поддерживать плагины-расширения. Или наоборот, для приложения может быть разработана развитая система плагинов с автоматическим контролем и обновлением версий, с удобным <<магазином>> расширений, но поддержка автоматизации может отсутствовать.

Как будет показано далее, некоторые существующие решения для интеграции возможностей автоматизации и расширяемости поддерживают сразу оба механизма, некоторые --- только что-то одно. Этому есть объяснение. В некоторых случаях нет необходимости реализовывать оба подхода. Этот вывод подтверждается практикой --- многие современные приложения поддерживают либо автоматизацию, либо расширения, и этого бывает достаточно. Но, как говорилось во введении, данные вопросы встают более остро в крупных программных комплексах, состоящих из множества компонентов. В таких ситуациях зачастую актуальны оба понятия. В связи с этим был сделан вывод о том, что понятия автоматизации и расширяемости можно рассматривать как одно целое. Более того, если это учесть при дизайне приложения, реализовывать поддержку этих механизмов также можно совместно и сделать их неразделимыми.




\TODO{Сюда теорминимум. Различия аддинов и скриптов, +/-}

\subsection{Формулировка задачи}

В рамках портирования с COM на .NET крупного приложения для управления инвестиционными портфелями потребовалось замена технологии VBA на аналогичною .NET совместимую технологию. Таким образом, потребовалось провести исследование рынка ПО и поиск продуктов с аналогичным или похожим функционалом.

Основные требования к программному продукту
Возможность создания расширений для конечных пользователей.
Простота использования.
Не требует SDK и другого ПО для создания расширений. Вся работа происходит в среде разработки (IDE).
Возможность работы как под x86, так и под x64 архитектурами.
Возможность отладки расширения.
Доступ расширения к объектам расширяемого приложения, реакция на его события.
При изменении удалении или добавлении объекта в хост приложении, расширение должно автоматически обновлять информацию о своем окружении.
Доступ к библиотеке классов .NET Framework.
Поддержка большого числа расширений и взаимодействие их друг с другом.
Возможность разрешения зависимостей между расширениями.
Удобные инструменты для написания кода (Аналогично IntelliSense).
Возможность сохранения расширений по усмотрению пользователя в базу данных, архив, папку и т. д.
Динамическая загрузка и выгрузка расширений из адресного пространства хост приложения.
Компиляция и перезагрузка расширения «на лету» (Не требуется перезапуск хост приложения)

\subsection{Применимость существующих решений}
\paragraph{VBA}
\paragraph{VSTA}
\paragraph{VSTO}
\subsection{Разработка новой платформы для решения поставленной задачи}

\pagebreak