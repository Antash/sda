\section{Теоретическое исследование}

\subsection{Расширяемость и автоматизация как единое понятие}
\label{sec:autom-and-ext-as-one-thing}

Во введении были даны определения расширяемости и автоматизации программного обеспечения. С одной стороны, эти два понятия имеют мало общего и могут существовать раздельно. Действительно, программный продукт может предоставлять возможность использовать скрипты для автоматизации, но не поддерживать плагины-расширения. Или наоборот, для приложения может быть разработана развитая система плагинов с автоматическим контролем и обновлением версий, с удобным <<магазином>> расширений, но поддержка автоматизации может отсутствовать.

Как будет показано далее, некоторые существующие решения для интеграции возможностей автоматизации и расширяемости поддерживают сразу оба механизма, некоторые --- только что-то одно. Этому есть объяснение. В некоторых случаях нет необходимости реализовывать оба подхода. Этот вывод подтверждается практикой --- многие современные приложения поддерживают либо автоматизацию, либо расширения, и этого бывает достаточно. Но, как говорилось во введении, данные вопросы встают более остро в крупных программных комплексах, состоящих из множества компонентов. В таких ситуациях зачастую актуальны оба понятия. В связи с этим был сделан вывод о том, что понятия автоматизации и расширяемости можно рассматривать как одно целое. Более того, если это учесть при дизайне приложения, реализовывать поддержку этих механизмов также можно совместно и сделать их неразделимыми.




Для написания пользовательских расширений могут использоваться как скрипты (в терминологии некоторых программ «макросы»)~\cite{automata-via-macros}, так и плагины (независимые модули, написанные на компилируемых языках)~\cite{addins-and-extensibility}.

\subsubsection{Скриптовый язык удобен в следующих случаях:}

\begin{enumerate}
\item Если нужно обеспечить программируемость без риска дестабилизировать систему. Так как, в отличие от плагинов, скрипты интерпретируются, а не компилируются, неправильно написанный скрипт выведет диагностическое сообщение, а не приведёт к системному краху;
\item Если важен выразительный код. Во-первых, чем сложнее система, тем больше кода приходится писать «потому, что это нужно. Во-вторых, в скриптовом языке может быть совсем другая концепция программирования, чем в основной программе — например, игра может быть монолитным однопоточным приложением, в то время как управляющие персонажами скрипты выполняются параллельно. В-третьих, скриптовый язык имеет собственный проблемно-ориентированный набор команд, и одна строка скрипта может делать то же, что несколько десятков строк на традиционном языке. Как следствие, на скриптовом языке может писать программист очень низкой квалификации — например, дизайнер своими руками, не полагаясь на программистов, может корректировать правила игры;
\item Если требуется кроссплатформенность. Хорошим примером является JavaScript — его исполняют браузеры под самыми разными ОС.
\end{enumerate}

\subsubsection{У плагинов же есть три важных преимущества:}

\begin{enumerate}
\item Готовые программы, оттранслированные в машинный код, выполняются значительно быстрее скриптов, которые интерпретируются из исходного кода динамически при каждом исполнении. Поэтому скриптовые языки не применяются для написания программ, требующих оптимальности и быстроты исполнения. Но из-за простоты они часто применяются для написания небольших, одноразовых («проблемных») программ;
\item Полный доступ к любому аппаратному обеспечению или ресурсу ОС (в скриптовом языке для этого должен существовать написанный на машинном коде API). Плагины, работающие с аппаратным обеспечением, традиционно называют драйверами;
\item Если предполагается интенсивный обмен данными между основной программой и пользовательским расширением, для плагина его обеспечить проще.
\end{enumerate}

Также в плане быстродействия скриптовые языки можно разделить на языки динамического разбора (sh) и предварительно компилируемые (perl). Языки динамического разбора считывают инструкции из файла программы минимально требующимися блоками, и исполняют эти блоки, не читая дальнейший код. Предкомпилируемые языки транслируют всю программу в байт-код и затем исполняют его. Некоторые скриптовые языки имеют возможность компиляции программы «на лету» в машинный код (т. н. JIT-компиляция).

\subsection{Формулировка задачи}

В рамках портирования с COM на .NET крупного приложения для управления инвестиционными портфелями потребовалось замена технологии VBA на аналогичною .NET совместимую технологию. Таким образом, потребовалось провести исследование рынка ПО и поиск продуктов с аналогичным или похожим функционалом.

\subsubsection{Основные требования к программному продукту}
\label{sec:requirements}
\begin{enumerate}
\item Возможность создания расширений для конечных пользователей;
\item Простота использования (Не требует SDK и другого ПО для создания расширений. Вся работа происходит в среде разработки (IDE));
\item Возможность работы как под x86, так и под x64 архитектурами;
\item Возможность отладки расширения;
\item Доступ расширения к объектам расширяемого приложения, реакция на его события;
\item При изменении удалении или добавлении объекта в хост приложении, расширение должно автоматически обновлять информацию о своем окружении;
\item Доступ к библиотеке классов .NET Framework;
\item Поддержка большого числа расширений и взаимодействие их друг с другом;
\item Возможность разрешения зависимостей между расширениями;
\item Удобные инструменты для написания кода (Аналогично инструментам IntelliSense в Microsoft Visual Studio);
\item Возможность сохранения расширений по усмотрению пользователя в базу данных, архив, папку и т. д.;
\item Динамическая загрузка и выгрузка расширений из адресного пространства хост приложения;
\item Компиляция и перезагрузка расширения «на лету» (Не требуется перезапуск хост приложения).
\end{enumerate}

\subsection{Применимость существующих решений}
\paragraph{VBA}
\paragraph{VSTA}
\paragraph{VSTO}



\pagebreak