\subsection{Применимость существующих решений}

% =======================================================================================================================================
% =======================================================================================================================================

\subsubsection{VBA}

Как уже обсуждалось ранее, реальный проект, в рамках которого и возникла задача разработки платформы для автоматизации и расширения приложений, подразумевал портирование устаревшего приложения на современные языки и платформы. В <<старом>> приложении для автоматизации и расширения как раз и использовался {\it VBA}. Данное решение обладало многими преимуществами:
\begin{itemize}
\item простота интеграции в основное приложение;
\item наличие удобной среды разработки;
\item наличие отладчика;
\item наличие инструментов для сохранения и загрузки проектов;
\item возможность использовать доступные в операционной системе {\it COM}  объекты и {\it ActiveX} компоненты;
\item простота реализации взаимодействия основного приложения и расширения.
\end{itemize}

Исходя из перечисленных особенностей, можно сделать вывод, что для <<старого>> приложение решение с {\it VBA} было оптимальным. Однако по ряду причин невозможно использовать {\it VBA} в современном приложении не {\it .NET}:

\begin{itemize}
   \item {\it VBA} не совместим с {\it .NET};
   \item нет поддержки 64-битных приложений;
   \item {\it VBA} является устаревшей и не поддерживается.
\end{itemize}

% =======================================================================================================================================
% =======================================================================================================================================

\subsubsection{VSTA}


% =======================================================================================================================================
\subsubsection{VSTO}
