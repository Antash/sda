\subsubsection{Visual Studio Tools for Applications}

{\it Visual Studio Tools for Applications (VSTA)} -- это набор инструментов, который независимые поставщики программного обеспечения могут использовать для добавления возможностей автоматизации и расширения в свои приложения. {\it VSTA} был объявлен {\it Microsoft} с выпуском интегрированной реды разработки {\it Visual Studio 2005}. {\it VSTA} основана на платформе {\it .NET Framework} и построена на той же архитектуре, что и {\it VSTO}, речь о которой пойдёт в следующем разделе. {\it VSTA} включает в себя интегрированную среду разработки, являющуюся упрощённой версией IDE {\it Visual Studio}. Основные языки программирования, используемые в {\it VSTA} -- это {\it Visual Basic .NET} и {\it C\#}, однако в общем случае может быть использован любой язык для платформы {\it .NET}. Благодаря тесной интеграции с {\it .NET} пользовательский код имеет доступ ко всей библиотеке классов этой платформы, что существенно расширяет спектр возможностей при написании расширений и макросов. Важной особенностью {\it VSTA} является поддержка 64-битной архитектуры. 

Платформа {\it VSTA} распространется бесплатно, однако разработчики, планирующие включать {\it VSTA} в состав коммерческого приложения, должны приобрести соответствующую лицензию.