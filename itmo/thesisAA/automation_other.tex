\subsubsection{Другие разработки}

Помимо универсальных платформ для автоматизации, рассмотренных выше, существует множество скриптовых языков, которые могут быть использованы для создания собственной платформы, решающей задачи автоматизации ПО. Некоторые из этих языков (как правило, речь идет о диалектах языка, нацеленных на эффективную работу с одним из программных продуктов) разрабатывались специально для конкретной задачи по автоматизации приложения. Это позволило глубоко интегрировать скриптовый язык в архитектуру приложения, которое он расширяет. Яркими примерами таких языков являются диалекты языка Lisp. AutoLisp с 1986 года применяется в AutoCAD и неразрывно с ним связан, Emacs Lisp используется в текстовых редакторах GNU Emacs и XEmacs. Особенностью таких узкоспециализированных скриптовых языков является глубокая интеграция с приложением и сложность в освоении. Они хорошо подходят для решения задач, специфичных для конкретного приложения, но их использование как универсального инструмента для автоматизации ПО невозможно.

Также существует несколько скриптовых языков, совместимых с платформой .NET. Наиболее перспективным и популярным из них является IronPython, на примере которого и будет рассмотрена концепция использования скриптового языка для автоматизации основного приложения.


