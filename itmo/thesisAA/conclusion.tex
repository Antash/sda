\section{Сравнение с аналогами}

Единственным непосредственным аналогом разработанной платформы является VSTA. (речь идет о ПО для платформы .NET)  Сравнение будет происходить основываясь на опыте применения технологии VSTA в реальном проекте (см раздел \ref{sec:use_exis_techn}). Кроме того, в рамках данной работы необходимо оценить успешнось разработанной платформы, как замены устаревшего инструментария VBA, так как один из сценариев применения платформы - замена VBA при портировании COM/ActveX-приложений, использующих VBA, на платформу .NET.

\subsubsection{Методика сравнительного анализа}

Рассматриваемые плаформы поддержки расширений можно представить как совокупность следующих компонент:

\begin{itemize}
   \item среда разработки расширений;
   \item пользовательский интерфайс (инструментарий для работы с расширениями);
   \item ядро, обеспечивающее взаимодействие приложения с расширениями.
\end{itemize}

Для получения адекватных результатов сравнительный анализ необходимо проводить по каждой из этих компонент. Так как сформулировать критерии для получения точных количественных оценок при сравнении, к примеру, удобства пользовательского интерфейса, не представляется возможным, все выводы будут следовать по большей части личного опыта использования тех или иных продуктов и инструментов, а так же из отзывов других разработчиков или пользователей этих продуктов.

Кроме перечисленных компонент системы немаловажным фактором в выборе того или иного продукта будет являться простота его использования. В конце раздела будет проведено сравнение сложности интеграции продуктов в приложение.

\subsubsection{Среда разработки}

VSTA Studio является логичным развитием VB6 Studio, используемой как редактор VBA-кода. Большинство основных элементов управления сргуппированы аналогично и выполняют аналогичные операции в этих IDE. Однако, VSTA имеет более <<продвинутый>> визуальный редактор форм, а так же новые инструменты IntelliSense, увеличивающие эффективность разработки. Используемая в разработанной платформе IDE SharpDevelop имеет похожий функционал, однако набор возможностей редактора кода не столь широк, как в VSTA Studio. В этом мы убедились ранее, реализовав генератор сигнатур обработчиков событий, отсутствующий в SharpDevelop по умолчанию. (см. раздел \ref{sec:ehsg}) Так же, из-за того что использование этой IDE не подразумевает сценария использования как интегрированой среды разработки расширений, проявилось множество проблем, требующих разрешения. (поднобнее про эти проблемы и методы их решения см. раздел \ref{sec:dev-details}) Не смотря на отсутствие некоторых функций в базовом установочном пакете SharpDevelop, его функционал может быть расширен засчет механизма плагинов, либо интегрированной технологии SDA.

Средства разработки компании Microsoft предоставляют более шировий набор возможностей <<из коробки>>, однако используемая в разработанной платформе IDE является более гибким и масштабируемым решением. Кроме того, SharpDevelop является полноценной IDE для разработки программ на большом числе .NET-совместимых языков программирования, в то время как продукты Microsoft поддерживают только Visual Basic и имеют менее богатый встроенный инструментарий для разработчика.

\subsubsection{Пользовательский интерфейс}

Как таковые, графические средства для проведения операций с расширенями продукты Microsoft не предоставляют. То есть разработчику программного обеспечения необходимо реализовать свой собственный инструментарий в соответствии с поставлеными целями. В разработанную платформу интегрирован пользовательский интерфейс, реализованный на Windows Forms, предоставляющий основные инструменты для управления расширениями. (см. раздел \ref{sec:macro-gui})

Преимущества такого решения очевидны: разработчику программного обеспечения достаточно встроить готовые компоненты в свое приложение чтобы получить полнофункциональную систему боддержки расширений. Из недостатков стоит отметить неприемлемость такого решения в случае, если втроенные в платформу элементы нарушают целостность визуального оформления приложения. В этом случае имеет смысл использовать один из паттернов проектирования для предоставления разработчику возможность самому реализовать представления компонент графического интерфейса в соответствии со стилем разрабатываемого приложения.

\subsubsection{Взаимодействие расширения и приложения}

В этом аспекте не имеет смысла сравнивать разработанную платформу с VBA из-за принципиального отличия модели взаимодействия COM-компонент и .NET-библиотек между собой. В свою очередь, используя VSTA можно выбрать различные варианты работы с ним. В одном из случаев взаимодействие расширения и приложения будет происходить через System.Addin, особенности этого подхода были подробно описаны в разделе \ref{sec:system_addin}. В разрабатываемой платформе был выбран .NET Reflection в качестве основы для организации взаимодействия. По большому счету это и евляется основным отличием разработанной платформы от VSTA.

Эта часть платформы хоть и играет, пожалуй, самую важную роль, но скрыта от пользовательских глаз. Поэтому, если все пользовательские сценарии реализованы верно и не возникает проблем со стабильностью и производительностью, по большому счету не имеет смысла утверждать что тот или иной подход оказывается лучше или хуже. Минусом исполизуемого подхода (см. раздел \ref{sec:extention_interaction}) является тот факт, что сборки расширения остаются в домене приложения, то есть в памяти, до завершения работы с ним. Это сложно назвать серьезным недостатком, так как эти <<паразитные>> сборки полностью изолируются и не могут повлиять на работу приложения. Более того, их размер ничтожно мал по сравнению с доступными объемами памяти, иварьируется от десятков до сотен килобайт, в зависимости от объема кода расширения. В исключительных случаях, если расширение перегружено разнообразными ресурсами, размер может доходить до нескольких мегабайт. В конце-концов, так как проблема <<паразитных>> сборок имеет место только в случае активного использования отладчика расширения и редактирования его кода, она не будет мешать работе с уже готовыми расширениями.

\subsubsection{Интеграция}

Основным преимуществом разработаного решения является относительная простота его интеграции как в существующее приложение, так и на этапе разработки нового приложения. Простота достигается благодаря использованию разработанного механизма взаимодействия расширения и приложения, в основе которого лежит .NET Reflection. Подробнее про особенности реализованного механизма написано в разделе \ref{sec:extention_interaction}. Для предоставления доступа к объектам программисту требуется всего лишь реализовать этими объектами интерфейс, используемый модулем интеграции расширений разработанной платформы. В то же время, интеграция рассмотренных в обзоре (см. раздел \ref{sec:extention_interaction}) требует реализации сложных и громоздких протоколов взаимодействия на основе контрактов, а так же накладывает некоторые условия на архитектуру разрабатываемого приложения. Конечно, таким образом достигается изоляция расширения от приложения, но при этом интеграция сильно усложнена.

Так же стоит отметить наличие в разработанной платформе средств для управления расширениями, которые могут быть легко добавлены полностью или частично в целевое приложение на этапе интеграции. Если же такой вариант разработчика не устроит, он может реализовать собственные визуальные компоненты, однако, их придется интегрировать в платформу поддержки расширений, на что будет потрачено дополнительное время.

\subsubsection{Выводы}



\pagebreak


\section{Перспективы развития платформы}

Несмотря на то, что основные требования к платформе были удовлетворены, существует возможность реализовать дополнительные инструменты и расширить функциональность платформы для обеспечения большего удобства и достижения максимальной эффективности ее использования. Также, реализация некоторых дополнительных функций может способствовать расширению потенциальной аудитории пользователей данной платформы. На взгляд автора, у данной разработки существует несколько возможных направлений развития:

\begin{itemize}
   \item развитие и реализация инструментов для упрощения и частичной автоматизации интеграции платформы;
   \item реализация дополнительных инструментов для обеспечения эффективности и упрощения процесса разработки расширений на базе данной платформы;
   \item улучшение механизмов управления расширениями; распространения расширений;
   \item своевременная реализация поддержки новых версий SharpDevelop и .NET Framevork.
\end{itemize}

Рассмотрим некоторые возможности развития платфоры более подробно.

\subsubsection{Инструменты интеграции}

Для упрощения задачи интеграции платформы в готовое приложение (а именно такую задачу призвана решать разработанная платформа) имеет смысл продумать и реализовать набор инструментов для анализа кода приложения, выделения объектов, к которым хочется предоставить доступ расширению и автоматической генерации кода, реализующего нужные ядру платформы интерфейсы взаимодействия расширение-приложение для этих объектов. Эти инструменты также будут полезны и при реализации нового приложения, использующего представленную в работе платформу, так как позволят автоматизировать процесс.

\subsubsection{Инструменты управления расширениями}

Так как один из основных сценариев использования платформы --- редактирование кода расширения (правка ошибок, реализация новой фанкциональности, и т. д.), имеет смысл создать инструменты для коллективной разработки расширений, например реализовать возможность подключения к репозиторию, а также возможность отправки отчета об ошибке в расширении (возможно, содержащего и исправление этой самой ошибки) разработчику.

Также не лишней будет проработка системы распространения расширений для каждого конкретного приложения, на подобие популярных ныне AppStore и Play market.

\subsubsection{Обеспечение эффективности разработки}

Возможно возникновение ранее не предусмотренных сценариев использования SharpDevelop, когда его встроенных возможностей будет не хватать для обеспечения максимальной эффективности процесса разработки расширения. В этом случае может понадобиться реализация плагинов и/или других модулей для добавления интересующей функциональности, как это было сделано в разделе ~\ref{sec:ehsg}).

\subsubsection{Взгляд в будущее}

Помимо рассмотренных выше перспектив, можно предположить теоретически возможные для данного продукта сценарии использования. Например, стоит рассмотреть возможность понижения <<порога вхождения>> для конечного пользователя. Конечно, для реализации простых макросов пользователю не требуются глубокие знания программирования и владение множеством технологий и инструментов. Однако базовые знания конструкций языка и возможностей самой платформы все еще остаются необходимы. Более того, каждое приложение с поддержкой пользовательских макросов требует некоторого время на обучение пользователя работе с необходимым  инструментарием. Этот фактор являются сдерживающим для многих начинающих пользователей, работа которых могла бы стать куда более эффективной с использованием возможностей автоматизации. Решением этой проблемы может стать <<визуальное программирование>>. Уже довольно давно появились и с успехом применяются средства графического программирования, такие как {\it JMCAD}, {\it LabVIEW}, {\it HiAsm} и другие.

Отдельного внимания заслуживает программа {\it Automator}, разработанная компанией Apple. Она позволет по принципу drag-and-drop создавать скрипты для автоматического выполнения различных действий. В Automator можно использовать большое количество готовых программных блоков, выполняющих действия с использованием таких программ, как Finder, Safari, iCal, Address Book и т. д. Он позволяет также использовать и сторонние программы. Несмотря на то, что Automator использует AppleScript и/или Cocoa, для использования программы не требуется знаний этих языков, все действия выполняются полностью в графической среде, хотя существуют блоки для вставки, позволяющие выполнить код на AppleScript или в shell-среде. Общий принцип действия достаточно прост: выходные параметры одного действия являются входными параметрами следующего, действия выполняются поочерёдно, при этом также имеются способы как повторов действий, так и их зацикливания.~\cite{automator-website} Принципы, лежащие в основе программы {\it Automator}, можно применить для создания так называемого <<конструктора макросов>>, который, являясь частью разрабатываемой платформы, может сильно повлиять на рынок средств автоматизации программных продуктов.

Таким образом, данную работу также можно рассматривать как некоторый <<задел>> для будущих разработок в этом направлении.

\pagebreak


\setcounter{secnumdepth}{0}
\section{Заключение}
\setcounter{secnumdepth}{2}

В данной работе затронута одна из проблем, возникающих при разработке современных программных комплексов: проблема автоматизации и расширения ПО. Большинство современных крупных приложений так или иначе поддерживают автоматизацию либо расширение. В некоторых ситуациях для решения этой задачи используются какие-нибудь готовые разработки, в некоторых случаях реализуется какой-то специфичный механизм. Существует несколько готовых программных решений для интеграции возможностей автоматизации и расширения в приложения. В работе рассмотрены наиболее популярные из них. 

Необходимость интеграции возможностей автоматизации и расширения в приложение возникла на реальном коммерческом проекте. В рамках проекта необходимо было портировать приложение, разработанное с использованием устаревших технологий, на современные платформы. Автоматизация и расширение достигались в приложении за счёт технологии {\it Visual Basic for Applications}, которая, хоть и является на настоящий момент устаревшей, весьма успешно решала поставленную задачу. Найти подходящую замену для {\tt .NET} оказалось непросто. Наилучшим кандидатом казалась платформа {\it Visual Studio Tools for Applications}, которая и была внедрена в разрабатываемое приложение. Однако как на этапе разработки, так и не этапе тестирования возникло множество проблем. Более того, на момент окончания разработки выяснилось, что {\it VSTA} больше не поддерживается и лицензию на неё приобрести невозможно.

В результате было принято решение разрабатывать новую платформу, позволяющую интегрировать возможности расширения и автоматизации в приложения. При разработке платформы учитывались результаты исследования существующих решений, а также опыт внедрения одного из них на реальном проекте, разрабатываемом в компании First Line Software.

Разработанная платформа отвечает всем поставленным требованиям. Она нацелена на упрощение процесса интеграции возможностей автоматизации и расширения в приложения. Помимо модулей, встраиваемых в приложения, платформа содержит ряд утилит для анализа и генерации кода, что ускоряет процесс интеграции. Созданная платформа была внедрена в проект компании First Line Software. Результаты работы были высоко оценены заказчиками.

Разработанная платформа сравнивалась по основным критериям с существующими решениями, и в результате анализа был сделан вывод о том, что она действительно имеет ряд преимуществ перед существующими разработками.

\pagebreak