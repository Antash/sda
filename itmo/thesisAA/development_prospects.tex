\section{Перспективы развития платформы}

Несмотря на то, что основные требования к платформе были удовлетворены, существует возможность реализовать дополнительные инструменты и расширить функциональность платформы для обеспечения большего удобства и достижения максимальной эффективности ее использования. Также, реализация некоторых дополнительных функций может способствовать расширению потенциальной аудитории пользователей данной платформы. На взгляд автора, у данной разработки существует несколько возможных направлений развития:

\begin{itemize}
   \item развитие и реализация инструментов для упрощения и частичной автоматизации интеграции платформы;
   \item реализация дополнительных инструментов для обеспечения эффективности и упрощения процесса разработки расширений на базе данной платформы;
   \item улучшение механизмов управления расширениями; распространения расширений;
   \item своевременная реализация поддержки новых версий SharpDevelop и .NET Framevork.
\end{itemize}

Рассмотрим некоторые возможности развития платфоры более подробно.

\subsubsection{Инструменты интеграции}

Для упрощения задачи интеграции платформы в готовое приложение (а именно такую задачу призвана решать разработанная платформа) имеет смысл продумать и реализовать набор инструментов для анализа кода приложения, выделения объектов, к которым хочется предоставить доступ расширению и автоматической генерации кода, реализующего нужные ядру платформы интерфейсы взаимодействия расширение-приложение для этих объектов. Эти инструменты также будут полезны и при реализации нового приложения, использующего представленную в работе платформу, так как позволят автоматизировать процесс.

\subsubsection{Инструменты управления расширениями}

Так как один из основных сценариев использования платформы --- редактирование кода расширения (правка ошибок, реализация новой фанкциональности, и т. д.), имеет смысл создать инструменты для коллективной разработки расширений, например реализовать возможность подключения к репозиторию, а также возможность отправки отчета об ошибке в расширении (возможно, содержащего и исправление этой самой ошибки) разработчику.

Также не лишней будет проработка системы распространения расширений для каждого конкретного приложения, на подобие популярных ныне AppStore и Play market.

\subsubsection{Обеспечение эффективности разработки}

Возможно возникновение ранее не предусмотренных сценариев использования SharpDevelop, когда его встроенных возможностей будет не хватать для обеспечения максимальной эффективности процесса разработки расширения. В этом случае может понадобиться реализация плагинов и/или других модулей для добавления интересующей функциональности, как это было сделано в разделе ~\ref{sec:ehsg}).

\subsubsection{Взгляд в будущее}

Помимо рассмотренных выше перспектив, можно предположить теоретически возможные для данного продукта сценарии использования. Например, стоит рассмотреть возможность понижения <<порога вхождения>> для конечного пользователя. Конечно, для реализации простых макросов пользователю не требуются глубокие знания программирования и владение множеством технологий и инструментов. Однако базовые знания конструкций языка и возможностей самой платформы все еще остаются необходимы. Более того, каждое приложение с поддержкой пользовательских макросов требует некоторого время на обучение пользователя работе с необходимым  инструментарием. Этот фактор являются сдерживающим для многих начинающих пользователей, работа которых могла бы стать куда более эффективной с использованием возможностей автоматизации. Решением этой проблемы может стать <<визуальное программирование>>. Уже довольно давно появились и с успехом применяются средства графического программирования, такие как {\it JMCAD}, {\it LabVIEW}, {\it HiAsm} и другие.

Отдельного внимания заслуживает программа {\it Automator}, разработанная компанией Apple. Она позволет по принципу drag-and-drop создавать скрипты для автоматического выполнения различных действий. В Automator можно использовать большое количество готовых программных блоков, выполняющих действия с использованием таких программ, как Finder, Safari, iCal, Address Book и т. д. Он позволяет также использовать и сторонние программы. Несмотря на то, что Automator использует AppleScript и/или Cocoa, для использования программы не требуется знаний этих языков, все действия выполняются полностью в графической среде, хотя существуют блоки для вставки, позволяющие выполнить код на AppleScript или в shell-среде. Общий принцип действия достаточно прост: выходные параметры одного действия являются входными параметрами следующего, действия выполняются поочерёдно, при этом также имеются способы как повторов действий, так и их зацикливания.~\cite{automator-website} Принципы, лежащие в основе программы {\it Automator}, можно применить для создания так называемого <<конструктора макросов>>, который, являясь частью разрабатываемой платформы, может сильно повлиять на рынок средств автоматизации программных продуктов.

Таким образом, данную работу также можно рассматривать как некоторый <<задел>> для будущих разработок в этом направлении.

\pagebreak
