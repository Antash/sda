\subsubsection{Другие разработки}

Помимо проектов, разрабатываемых и поддерживаемых коммерческими компаниями и сообществом, существуют независимые проекты, созданные с целью обучения технологиям, демонстрации возможностей тех или иных библиотек. Эти проекты интересны в первую очередь для изучения их внутренней архитектуры, так как они имеют открытый исходный код. Применение их для реального проекта требует значительных доработок.

Отдельного упоминания заслуживает технология {\it SDA (SharpDevelop  for applications)}. Это набор инструментов, позволяющий построить собственное приложение на основе ядра SharpDevelop. Это приложение может получить <<в наследство>> многие возможности самого SharpDevelop, в том числе и возможность использования плагинов (SharpDevelop Addins).

Всерьез рассматривать эту технологию, как готовую к решению задачи построения расширяемых приложений нельзя, так как свобода разработчика приложения на основе ядра SharpDevelop будет ограничена архитектурными особенностями этого ядра. Однако, SDA интересна тем, что позволяет программно управлять и самой IDE SharpDevelop, которая в свою очередь может быть интегрирована в приложение как встроенная среда разработки расширений или макросов.