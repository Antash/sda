\subsection{Межпроцессное взаимодействие}
\label{sec:ipc}

В каждой операционной системе предоставляются различные способы межпроцессного взаимодействия. Ниже перечислены наиболее популярные из них:
\begin{itemize}
   \item использование доступа к общему файлу;
   \item использование сигналов;
   \item сокеты;
   \item именованные каналы;
   \item разделяемая память;
   \item очереди сообщений.
\end{itemize}

Некоторые способы можно реализовывать вручную, для некоторых есть специальные функции в API операционной системы. Однако более высокоуровневые библиотеки и платформы, такие как {\it .NET}, предоставляют обобщённые и более универсальные механизмы для межпроцессного взаимодействия. Ниже будет рассмотрен один из них.

{\it WCF (Windows Communication Foundation)} --- программный фреймворк, используемый для обмена данными между приложениями, входящими в состав {\it .NET Framework}~\cite{wcf-unleashed}. {\it WCF} делает возможным построение безопасных и надёжных транзакционных систем через упрощённую унифицированную программную модель межплатформенного взаимодействия. Если не вдаваться в подробности, можно сказать, что {\it WCF} --- некая абстракция, обёртка, которая скрывает фактические детали реализации межпроцессного взаимодействия, которое на низком уровне будет реализовано за счёт одного из вышеперечисленных способов. 

Для осуществления межпроцессного должен быть написан интерфейс, который будет реализован в одном из процессов. Процесс, реализующий интерфейс, предоставляет сервис (службу), к которой может подключиться другой процесс, удалённо вызывая методы интерфейса~\cite{wcf-services}. Для обратного взаимодействия можно использовать callback-вызовы, либо создавать и реализовывать ещё один интерфейс и открывать второй канал для взаимодействия в обратную сторону. 

В разрабатываемой платформе межпроцессное взаимодействие нужно для:
\begin{itemize}
 \item управления средой разработки;
 \item передачи дополнительных данных среде разработки;
 \item передача событий среды основному приложению.
\end{itemize}


\pagebreak
