\documentclass[12pt,a4paper,oneside]{book}

\usepackage[utf8]{inputenc}
\usepackage{ucs}
\usepackage[english,russian]{babel}
\usepackage{extsizes} % чтобы можно было использовать шрифты больше, чем 12pt
\usepackage{pscyr} % красивые русскоязычные шрифты (мануал по установке тут: http://ishutov.blogspot.com/2011/02/miktex-29-pscyr-04d.html)
\usepackage{xcolor}
\usepackage{listings}
\usepackage{graphicx}
\usepackage{url}
\usepackage{lscape}

\usepackage{indentfirst}

% для математических символов
\usepackage{amsmath, amsthm, amssymb}

% для построения диаграмм
%\usepackage{tikz}
%\usetikzlibrary{positioning,arrows,shapes,shadows}

\lstset{inputencoding=utf8x}
\lstset{language=[Sharp]C}

\graphicspath{{pics/}}

% поля (в соответствии с Требованиями Оформления Магистерских Работ)
% по умолчанию левое и верхнее и левое поле = 1дюйм (2.54 см). Это соответствует требованиям.
\oddsidemargin = 0mm
\topmargin = -15mm % вычитается размер колонтитула.
\textwidth = 175mm
\textheight = 247mm

% абзацный отступ
\parindent = 12mm

\sloppy
\makeatletter
\renewcommand{\baselinestretch}{1.5} % межстрочный интервал

% чтобы каждая новая глава начиналась с новой страницы
\let\stdsection\section
\renewcommand\section{\newpage\stdsection}

\pagestyle{plain}

%\newcounter{appendix_number}
%\newcommand{\Appendix}[1]{\addtocounter{appendix_number}{1} \section{Приложение \arabic{appendix_number}. #1}}

% замена в подписи к рисунку разделителя ":" на "." (Рис. 2: Рисунок   станет   Рис. 2. Рисунок.) - - - - - - - - -
\renewcommand{\@makecaption}[2]{%
\vspace{\abovecaptionskip}%
\sbox{\@tempboxa}{#1: #2}
\ifdim \wd\@tempboxa >\hsize
#1: #2\par
\else
\global\@minipagefalse
\hbox to \hsize {\hfil #1. #2\hfil}%
\fi
\vspace{\belowcaptionskip}}
% - - - - - - - - - - - - - - - - - - - - - - - - - - - - - - - - - - - - - - - - - - - - - - - - - - - - - - - - - 


% - - - - - - - - - - - - - - - - - - - - - - - - - - - - - - - - - - - - - - - - - - - - - - - - - - - - - - - - - 
% оформление заголовков глав
 \def\onelineskip{\normalsize\vskip\baselineskip}

 \renewcommand{\@makechapterhead}[1]{%
 {\raggedright
 \parindent=0cm\hangafter=1%
 \newbox\numberbox
 \setbox\numberbox\hbox{\normalfont\LARGE\textar{\bfseries\thechapter.\hskip0.5em}}%
 \hangindent=\wd\numberbox %
 \normalfont\LARGE\textar{\@chapapp{}\bfseries
 \unhbox\numberbox #1}\par
 \nopagebreak
 \onelineskip
 }}

 \renewcommand{\@makeschapterhead}[1]{%
 {\parindent=0pt \raggedright
 \normalfont\LARGE\textar{\bfseries #1}\par
 \nopagebreak
 \onelineskip
 }}
% - - - - - - - - - - - - - - - - - - - - - - - - - - - - - - - - - - - - - - - - - - - - - - - - - - - - - - - - - 






\begin{document}
    %\renewcommand{\@oddhead}{\hfill \thepage \hfill} % номера страниц по центру сверху
    \renewcommand{\thesection}{\arabic{section}}
    \renewcommand{\bibname}{Источники}
    
    \newcommand{\TODO}[1]{\fcolorbox{red}{red}{TODO :: #1}}
    \newcommand{\Definition}[1]{{\bf Определение. } #1 \par} % TODO :: Добавить счётчик
   
    \setcounter{page}{6}   % номер первой страницы
	
	Антон Ашмарин, СПбГПУ, 2012 г.
    
    % Аннотация
	%\setcounter{secnumdepth}{0}
\section*{Аннотация}
\setcounter{secnumdepth}{2}

При разработке современных программных комплексов зачастую возникают задачи, связанные с поддержкой функций расширения и автоматизации программного обеспечения. Это необходимо для реализации возможности конфигурирования, настройки, переопределения поведения, а также реализации недостающего функционала приложения конечным пользователем. Чаще всего реализация данных возможностей достигается за счёт функций автоматического обновления программного обеспечения, поддержки плагинов сторонних производителей, наличия SDK (Software Development Kit) для разработки расширений, поддержки скриптов. С точки зрения числа предоставляемых возможностей и гибкости наибольший интерес представляет последний вариант. В зависимости от специфики разрабатываемого ПО могут быть различные сценарии использования скриптов: простейшее конфигурирование приложения конечным пользователем, разработка дополнений, расширяющих возможности приложения, распространение скриптов, разработанных сторонними производителями. В настоящее время многие разработчики предоставляют возможность использовать скрипты в своих приложениях. Помимо этого, существуют специальные технологии и платформы для интеграции скриптовых движков в приложения: VBA (Visual Basic for Applications), VSTA (Visual Studio Tools for Applications), VSTO (Visual Studio Tools for Office), IronPython, Game Maker Language) и другие. Основная проблема заклюается в том, чаще всего в каждом семействе программных продуктов разрабатываются собственные инструменты для интеграции поддержки скриптов в ПО. Существующие <<универсальные>> средства для решения этой задачи оказываются неэффективными и неудобными для использования на реальных проектах.

Целью данной работы является исследование существующих решений в области расширения и автоматизации ПО, а также создание универсальной платформы, включающей в себя инструменты для внедрения возможностей автоматизации и расширения в разрабатываемые программные продукты. Помимо самого механизма скриптов, разрабатываются утилиты для автоматического анализа кода основной программы и генерации вспомогательного кода. Платформа включает в себя механизм для отладки скриптов. Для достижения наибольшей эффективности и охвата максимального числа сценариев использования в предложенном подходе объединяются возможности использования как интерпретируемых, так и компилируемых языков. 

\pagebreak
	
	\tableofcontents
    \pagebreak
	
	% ---------  Основные главы -----------------------------------
	\setcounter{secnumdepth}{0} % за счёт этого "Введение" будет без номера
\section{Введение}
\setcounter{secnumdepth}{2}

Современное программное обеспечение (ПО) зачастую представляет из себя крупные программные комплексы, состоящие из множества различных компонентов. В процессе эксплуатации ПО одни компоненты могут устаревать, в других будут обнаружены критические ошибки и уязвимости. Помимо этого некоторым пользователям может оказаться необходим какой-то функционал, отсутствующий в базовой версии. Другие пользователи могут захотеть сконфигурировать приложение под свои нужды, автоматизировать некоторые его части. Все эти проблемы ведут к изучению вопросов, связанных с автоматизацией и расширением программного обеспечения.

Под {\it автоматизаций} программного продукта понимается возможность формировывать, сохранять и воспроизводить в автоматическом режиме некоторую последовательность пользовательских действий. К примеру, в популярном текстовом редакторе {\tt Microsoft Word} для ускорения часто выполняемых операций редактирования или форматирования, а также для объединения нескольких команд или автоматизации обработки сложных последовательных действий в задачах можно записать макрос. В дальнейшем можно запускать этот макрос для повторения ранее записанных действий.

Под {\it расширением} программного приложения понимается возможность добавлять в существующий продукт новые модули, расширяющие возможности программы. Примером может служить популярный файловый менеджер {\tt Far}, в который включён текстовый редактор. Однако в базовой конфигурации в редакторе нет подсветки синтаксиса языков программирования, в то время как многие разработчики используют его для написания кода. Недостоющий функционал добавляется за счёт планига (расширения) {\tt Far Colorer}. 

Программное приложение, в дизайне которого предусмотрены возможности автоматизации и расширения, предоставет целый набор дополнительных преимуществ и для разработчика основного приложения, и для сторонних разработчиков, и для конечного пользователя:
\begin{itemize}
\item конфигурирование приложения конечным ползователем
\item выпуск обновлённых и исправленных компонентов ПО
\item разработка дополнений сторонними программистами
\item автоматизация часто повторяющейся последовательности действий
\item дополнение недостоющего функционала за счёт написания скриптов
\end{itemize}

Раньше эти преимущества достигались за счёт периодических выпусков новых версий программного обеспечения. Это было не очень удобно, так как новые версию появлялись не так часто, и не было средств для решения проблем, содержащихся в текущей версии продукта. Был компромиссный вариант - выпуск пакетов исправления ({\tt Service Pack}) и пакетов новых возможностей ({\tt Feature pack}). 

Позже, с массовым распространением Интернета, этот подход был заменён автоматическим обнавлением. Суть осталась та же, только процесс установки пакетов исправлений и дополнений автоматически загружался с веб-сайта разработчика и автоматически устанавливался. 

Со временем, наряду с автоматическим обновлением, разработчики программного обеспечения стали добавлять в свои приложения возможности автоматизации и расширения. Этот шаг добавил программным комплексам большую гибкость, а также дал дополнительные возожности сторонним разработчикам. 

\pagebreak
	\section{Постановка задачи}

Цель работы --- разработка и интеграция платформы для интергации поддержки возможностей расширения и автоматизации в .NET-приложения.

Эта задача возникла в рамках проекта по портированию приложения для управления инвестиционными портфелями с COM на платформу .NET. В исходном приложении для интеграции возможностей автоматизации и расширения использовался набор инструментов VBA, включающий в себя скриптовый язык VBA и встроенную среду разработки VisualBasic6. Так как VBA не совместима с платформой .NET, требовалась реализация платформы с аналогичными возможностями. 

Первоначальной задачей стал анализ исходного приложения, изучение сценариев его использования, в особенности функционала для создания и управления VBA-макросами. На основе сценариев использования были сформулированы требования к искомой платформе. Опираясь на эти требования, было проведено исследование рынка ПО и поиск продуктов с аналогичным или похожим функционалом, которое могло бы быть использовано в качестве полной или частичной замены VBA в портируемом приложении.



\pagebreak
	%\section{Требования к платформе}

Платформа должна решать две основные задачи:
\begin{itemize}
 \item автоматизация ПО;
 \item расширение ПО.
\end{itemize}

В приложении, которое было разработано с использованием {\it VBA}, обе задачи решались за счёт возможности написания макросов на {\it Visual Basic}. Одни и те же макросы, в зависимости от контекста и конкретной задачи, могли как автоматизировать некоторые действия, так и расширять возможности приложения за счёт добавления нового функционала. Не было какого-то конкретного деления на <<автоматизацию>> и <<расширение>>. Поддержка макросов была реализована как некий отдельный модуль, тесно взаимодействующий с основным приложеним, что было достаточно удобно и прозрачно для пользователя. Это и навело впервые на мысль рассматривать понятия автоматизации и расширения как одно целое (далее об этом речь пойдёт в разделе~\ref{sec:autom-and-ext-as-one-thing}).

\subsubsection{Сценарии использования}

\subsubsection{Дополнительные возможности}

\pagebreak

	\section{Существующие решения}

\subsection{Автоматизация}

\subsubsection{Visual Basic for Applications}

{\it Visual Basic for Applications} -- разработка компании {\it Microsoft}, совмещающая в себе реализацию языка программирования {\it Visual Basic} и соответствующую среду разработки~\cite{mastering-vba}. {\it VBA} является интерпретируемым языком: исходный код программы компилируется в промежуточный {\it P-код}, который впоследствии выполняется виртуальной машиной, являющейся частью основного приложения. {\it VBA} построен на технологии {\it COM}, благодаря чему {\it VBA-}код может использовать все доступные в операционной системе {\it COM}-объекты и компоненты {\it ActiveX}. Интегрированная среда разработки предоставляет удобный редактор кода с подсветкой синтаксиса и системой автодополнения кода, набор инструментов для удобной работы с проектами и файлами исходного кода, а также отладчик. Одно из важных достоинств {\it VBA} -- сравнительная лёгкость освоения.

Изначально {\it VBA} был встроен в семейство продуктов {\it Microsoft Office}. Из кода макроса на {\it VBA} пользователь может получить доступ ко всем функциям, доступным через графический интерфейс пользователя, а также к многочисленным функциям операционной системы и компонентов сторонних разработчиков.

Помимо {\it Microsoft Office} {\it VBA} встроен во многие программные продукты других разработчиков, к примеру,  {\it AutoCAD}, {\it SolidWorks}, {\it CorelDRAW}, {\it WordPerfect} и {\it ESRI ArcGIS}. Более того, {\it VBA} можно использовать при разработке любого {\it Windows}-приложения для внедрения в него возможности автоматизации. 

В настоящее время технология {\it VBA} считается устаревшей и более не поддерживается. В качестве замены {\it Microsoft} предлагает технологии {\it VSTA}~\cite{vsta-website} и {\it VSTO}~\cite{vsto-website}, речь о которых пойдёт в следующих разделах.
\subsubsection{Visual Studio Tools for Applications}

{\it Visual Studio Tools for Applications (VSTA)} -- это набор инструментов, который независимые поставщики программного обеспечения могут использовать для добавления возможностей автоматизации и расширения в свои приложения. {\it VSTA} был объявлен {\it Microsoft} с выпуском интегрированной реды разработки {\it Visual Studio 2005}. {\it VSTA} основана на платформе {\it .NET Framework} и построена на той же архитектуре, что и {\it VSTO}, речь о которой пойдёт в следующем разделе. {\it VSTA} включает в себя интегрированную среду разработки, являющуюся упрощённой версией IDE {\it Visual Studio}. Основные языки программирования, используемые в {\it VSTA} -- это {\it Visual Basic .NET} и {\it C\#}, однако в общем случае может быть использован любой язык для платформы {\it .NET}. Благодаря тесной интеграции с {\it .NET} пользовательский код имеет доступ ко всей библиотеке классов этой платформы, что существенно расширяет спектр возможностей при написании расширений и макросов. Важной особенностью {\it VSTA} является поддержка 64-битной архитектуры. 

Платформа {\it VSTA} распространется бесплатно, однако разработчики, планирующие включать {\it VSTA} в состав коммерческого приложения, должны приобрести соответствующую лицензию.
\subsubsection{Visual Studio Tools for Office}

{\it Visual Studio Tools for Office} -- это набор средств разработки, доступных в виде расширений {\it Visual Studio} (шаблонов проектов) и исполняющей среды, позволяющей  семейству продуктов {\it Microsoft Office} 2003 и более поздних версий загружать и выполнять пользовательский код {\it .NET} для расширения функциональности приложений {\it Office} и их автоматизации. {\it VSTO} пришла на замену технологии {\it VBA}, использующейся в более ранних версиях {\it Microsoft Office}. 

Расширения {\it VSTO} разрабатываются в интергированной среде {\it Visual Studio}. Пользовательский код имеет доступ ко всей библиотеке классов {\it .NET}. Как и в случае с {\it VBA}, макрос выполняется виртуальной машиной (в данном случае - CLR, Common Language Runtime), которая работает в рамках процесса основного приложения. Но в отличие от {\it VBA} код сохраняется в отдельной сборке {\it .NET}.
\subsubsection{Другие разработки}

Помимо универсальных платформ для автоматизации, рассмотренных выше, существует множество скриптовых языков, которые могут быть использованы для создания собственной платформы, решающей задачи автоматизации ПО. Некоторые из этих языков (как правило, речь идет о диалектах языка, нацеленных на эффективную работу с одним из программных продуктов) разрабатывались специально для конкретной задачи по автоматизации приложения. Это позволило глубоко интегрировать скриптовый язык в архитектуру приложения, которое он расширяет. Яркими примерами такогих языков являются диалекты языка Lisp. AutoLisp с 1986 года применяется в AutoCAD и неразрывно с ним связан, Emacs Lisp используется в текстовых редакторах GNU Emacs и XEmacs. Особенностью таких узкоспециализированных скриптовых языков является глубокая интеграция с приложением и сложность в освоении. Они хорошо подходят для решения задач, специфичных для конкретного приложения, но их использование как универсального инструмента для автоматизации ПО невозможно.

Так же существует несколько скриптовых языков, совместимых с платформой .NET. Наиболее перспективным и популярным из них является IronPython, на примере которого и будет рассмотрена концепция использования скриптового языка для автоматизации основного приложения.


\subsubsection{IronPython} % TODO :: перенести в отдельный файл?
{\it IronPython} -- реализация языка программирования {\it Python} для платформ {\it .NET Framework} и {\it Mono}. {\it IronPython} полностью написан на {\tt C\#}, и является транслятором компилирующего типа. Код скрипта, написанного на {\it IronPython}, может использоваться одним из следующих способов:
\begin{itemize}
 \item посредством компиляции в независимую сборку, которая в дальнейшем может быть загружена в приложение как зависимось или с помощью {\it .NET Reflection};
 \item посредством размещения {\it IronPython}-подсистемы в основном приложении и динамической трансляции кода.
\end{itemize}

Эта особенность {\it IronPython} очень важна в контексте задач, связанных с автоматизацией ПО.

На схеме ниже рассмотрен типичный способ использования {\it IronPython} с целью автоматизации приложения:

\TODO{ (dennis.yolkin) схема и её описание}.

\subsection{Расширение}

\subsubsection{MEF (Managed Extensibility Framework)}

{\it MEF} -- библиотека для создания расширяемых приложений. Она позволяет разработчикам программного обеспечения использовать расширения без необходимости дополнительного конфигурирования. Благодаря {\it MEF}, исходный код расширения полностью независим и не возникает проблем с сильной связностью компонентов приложения и расширения. С помощью {\it MEF} можно разрабатывать как расширения для конкретного приложения, так и универсальные расширения, которые могут быть интегрированы в самые различные продукты.

Обычно в основе любой системы плагинов лежит некий общий интерфейс, который должен быть реализован любым расширением. Интерфейс описывает способы взаимодействия основного приложения и расширения. Это накладывает дополнительное ограничение, добавляя лишнюю зависимость между приложением и плагином. Помимо этого, никак не регламентируется способ взаимодействия между различными плагинами. Ситуация, в которой один плагин зависит от набора других плагинов (и может функционировать только если разрешены все зависимости), в общем случае оказывается неразрешимой.

Подход, выбранный разработчиками {\it MEF} направлен в первую очередь на упрощение разрешения зависимостей между основным приложением и расширением и между различными расширениями.

Вместо необходимости явной регистрации доступных компонентов-расширений, {\it MEF} предоставляет возможность автоматически неявно обнаруживать подходящие расширения. Любое расширение в {\it MEF} декларативно описывает все свои зависимости (в терминах платформы - {\tt imports}) и предоставляемые возможности ({\tt exports}).

Такое решение решает проблемы с зависимостями, описанные ранее. Раз зависимости и предоставляемый функционал описывается декларативно, любое расширение может быть <<подключено>> к основному приложению в процессе выполнения (без перезапуска приложения и тем более без перекомпиляции исходного кода). Таким образом, нет жёстких зависимостей между расширением и расширяемым приложением, а также между плагинами. Нет необходимости описывать какие-либо параметры в конфигурационных файлах. Вся необходимая информация может быть получена из метаданных, доступных при загрузке сборки с расширением.

За счёт объявления списка зависимостей, ими легко управлять. Каждое расширение <<знает>>, какие расширения необходимы для его корректной работы. За счёт этого достигается простота взаимодействия между расширениями.

Раз {\it MEF-}расширение не требует наличия жёстких зависимостей от конкретного расширяемого приложения, автоматически достигается возможность использовать одно и то же расширение в различных приложениях без необходимости специального конфигурирования. Помимо этого, появляются дополнительные возможности для тестирования: для тестирования или исследования возможностей конкретного расширения не обязательно привязываться к какому-либо существующему программному продукту, достаточно разработать тестовое окружение, к которому можно будет подключить это расширение.

\subsubsection{AL Platform}

\TODO{тут только вода}
\TODO{из описания вообще непонятно, в чём фишка этой платформы}
\TODO{такое ощущение, что эта платформа заточена только на проталкивании GUI из плагина в хост}
\TODO{надо бы установить это чудо себе и разобраться с ним.}

{\it AL Platform} --- платформа для быстрой разработки гибких решений со сложным интерфейсом пользователя. {\it AL Platform} вводит концепцию абстрактного рабочего стола, и определяет новую методологию разработки --- <<Инструментально-ориентированное подход>>. Инструмент --- это независимый программный модуль, плагин, который размещается внутри других приложений. Идея абстрактного рабочего стола позволяет таким инструментам <<наполнять>> рабочее пространство приложения и его меню без перекомпиляции приложения или инструмента. {\it AL Platform} не являетя внешним самостоятельным приложением, это библиотека классов, поставляемая в виде набора {\tt .NET} сборок. 

{\it AL Platform} позволяет выстроить архитектуру приложения таким образом, чтобы оно состояло из интегрированных модулией (плагинов), которые в терминологии платформы называются инструментами. Каждый инструмент включает в себя некоторую бизнес-логику. В зависимости от решаемых задач, инструмент может использоваться в различных приложениях. Инструменты разрабатываются независимо различными командами разработчиков. Плагины могут быть собраны в разные {\tt .NET} сборки и размещены физически на разных компьютерах. {\it AL Platform} предоставляет возможность таким инструментам взаимодействовать друг с другом, минимизируя при этом накладные расходы.

С точки зрения программиста, для создания нового инструмента нужно разработать класс, унаследованный от {\tt Al.Application.Tool} или {\tt Al.DesktopApplication.DesktopTool}. Компонент приложения, отвечающий за работу с плагинами, называется менеджером инструментов. Менеджер инструментов определяет, как тот или иной инструмент будет отображаться в рабочем пространстве приложения, а также предоставляет слой инфраструктуры, необходимыйй для взаимодействия инструмента и приложения. 

Все возможности приложения на базе {\it AL Platform} конфигурируются с помощью {\tt XML} файла с настройками. За счёт этого достигается максимальная гибкость персонализации существующих приложений.

Важной особенностью платформы является возможность <<связывать>> инструменты между собой. Это означает, что инструмент может быть запушен в одном менеджере инструментов, а использован в другом.
\subsubsection{Plux.NET}

{\it Plux.NET} --- платформа для работы с плагинами для {\tt .NET}, позволяющая создавать расширяемые приложения, состоящие из <<ядра>>, содержащего минимальный набор функций и возможностей, и набора расширений~\cite{plux-website}. В терминологии {\it Plux.NET} ядро --- это основное приложение (host). В ядре содержится набор слотов. Слот определяется как некий контракт между хост-приложением и расширениями, которые могут <<вставляться>> в слот. Загрузка расширений происходит во время выполнения программы. 

В простейшем случае, слот определяет интерфейс, а также набор параметров с типами. Расширение предоставляет реализацию этого интерфейса вместе со списком значений требуемых параметров. Хост-приложение  будет использовать эти параметры для загрузки и интеграции расширения. Слоты и расширения описываются в виде атрибутов {\tt .NET}.

Для примера описанной концепции рассмотрим абстрактное приложение с графическим пользовательским интерфейсом, которое позволяет в зависимости от установленных расширений добавлять новые команды в меню. Основное приложение <<открывает>> слот, описывая интерфейс {\tt IMenuItem} и два параметра: {\tt Text} (название элемента меню) и {\tt Icon} (пиктограмма пункта меню). Расширение --- это класс, реализующий интерфейс {\tt IMenuItem}, а также значения параметров (к примеру, <<Print>> в качестве названия элемента меню и графический файл <<Print.jpg>> в качестве пиктограммы). На рисунке \ref{plux-scheme} показана схема взаимодействия основного приложения с открытым слотом и расширения:

\begin{figure}[!h]
    \centering
    \includegraphics[width=12cm]{plux1.jpg}
    \caption{Подключение расширения к слоту в платформе Plux.NET}
    \label{plux-scheme}
\end{figure}


Для каждого открытого слота платформа {\it Plux.NET} определяет доступные подходящие расширения, загружает их, присваивает значения параметрам и уведомляет хост-приложение о подключении расширения к слоту. После этого хост-приложение предпринимает действия по интеграции расширения (в примере с приложением с графическим пользовательским интерфейсом, добавляя пункты в меню и устанавливая обработчики событий).
\subsubsection{Mono.AddIns, System.AddIn}

Mono.Addins is a generic framework for creating extensible applications, and for creating add-ins which extend those applications. This framework has been designed to be useful for a wide range of applications: from simple applications with small extensibility needs, to complex applications which need support for large add-in structures.

The main features of Mono.Addins are:
Supports descriptions of add-ins using custom attributes (for simple and common extensions) or using XML manifests (for more complex extensibility needs).
Supports extension metadata and data-only extensions.
Support for add-in hierarchies, where add-ins may depend on other add-ins.
Lazy loading of add-ins.
Dynamic activation / deactivation of add-ins at run time.
Allows sharing add-in registries between applications, and defining arbitrary add-in locations.
Allows implementing extensible libraries.
Supports add-in localization.
Provides an API for accessing to add-in descriptions, which will allow building development and documentation tools for handling add-ins.
In addition to the basic add-in engine, it provides a Setup library to be used by applications which want to offer basic add-in management features to users, such as enabling/disabling add-ins, or installing add-ins from on-line repositories.

\subsubsection{Compact Plugs \& Compact Injection}

{\tt Compact Plugs (CP)} --- неблольшая платформа для добавление возможностей расширения в приложения {\tt .NET Framework} и {\tt .NET Compact Framework}. {\tt Compact Plugs} зависит от {\tt Compact Injection} ({\tt Depemdency Injection}-контейнер для {\tt .NET}). {\tt Compact Plugs} использует {\tt Compact Ingection} для внедрения результатов работы плагинов в основное приложение во время его выполнения. Одна из основных идей {\tt Compact Plugs} заключается в том, что любой компонент может быть использован как плагин для другого компонента, и для достижения этой цели практически не нужно писать код. 

\TODO{Какая-то хуйня и ничего непонятно. Может выкинуть этот раздел?}
\subsubsection{Другие разработки}

Помимо проектов, разрабатываемых и поддерживаемых коммерческими компаниями и сообществом, существуют независимые проекты, созданные с целью обучения технологиям, демонстрации возможностей тех или иных библиотек. Эти проекты интересны в первую очередь для изучения их внутренней архитектуры, так как они имеют открытый исходный код. Применение их для реального проекта требует значительных доработок.

Отдельного упоминания заслуживает технология {\it SDA (SharpDevelop  for applications)}. Это набор инструментов, позволяющий построить собственное приложение на основе ядра SharpDevelop. Это приложение может получить <<в наследство>> многие возможности самого SharpDevelop, в том числе и возможность использования плагинов (SharpDevelop Addins).

Всерьез рассматривать эту технологию, как готовую к решению задачи построения расширяемых приложений нельзя, так как свобода разработчика приложения на основе ядра SharpDevelop будет ограничена архитектурными особенностями этого ядра. Однако, SDA интересна тем, что позволяет программно управлять и самой IDE SharpDevelop, которая в свою очередь может быть интегрирована в приложение как встроенная среда разработки расширений или макросов.

\pagebreak
	\section{Теоретическое исследование}

\subsection{Расширяемость и автоматизация как единое понятие}
\label{sec:autom-and-ext-as-one-thing}

Во введении были даны определения расширяемости и автоматизации программного обеспечения. С одной стороны, эти два понятия имеют мало общего и могут существовать раздельно. Действительно, программный продукт может предоставлять возможность использовать скрипты для автоматизации, но не поддерживать плагины-расширения. Или наоборот, для приложения может быть разработана развитая система плагинов с автоматическим контролем и обновлением версий, с удобным <<магазином>> расширений, но поддержка автоматизации может отсутствовать.

Как будет показано далее, некоторые существующие решения для интеграции возможностей автоматизации и расширяемости поддерживают сразу оба механизма, некоторые --- только что-то одно. Этому есть объяснение. В некоторых случаях нет необходимости реализовывать оба подхода. Этот вывод подтверждается практикой --- многие современные приложения поддерживают либо автоматизацию, либо расширения, и этого бывает достаточно. Но, как говорилось во введении, данные вопросы встают более остро в крупных программных комплексах, состоящих из множества компонентов. В таких ситуациях зачастую актуальны оба понятия. В связи с этим был сделан вывод о том, что понятия автоматизации и расширяемости можно рассматривать как одно целое. Более того, если это учесть при дизайне приложения, реализовывать поддержку этих механизмов также можно совместно и сделать их неразделимыми.




\TODO{Сюда теорминимум. Различия аддинов и скриптов, +/-}

\subsection{Формулировка задачи}
\subsection{Применимость существующих решений}
\paragraph{VBA}
\paragraph{VSTA}
\paragraph{VSTO}
\subsection{Разработка новой платформы для решения поставленной задачи}

\pagebreak
	
\section{Разработка платформы}

Ранее (см. раздел~\ref{sec:requirements}) уже были сформулированы основные требования к разрабатываемой платформе. При проектировании архитектуры платформы должны учитываться эти требования, преимущества и недостатки рассмотренных существующих решений, а также опыт внедрения одного из таких решений на реальном проекте (см. раздел~\ref{sec:use-itm-vsta}).

Вполне логично разделить систему на два компонента:
\begin{itemize}
 \item интерактивная среда разработки, содержащая в себе текстовый редактор, средство для управления проектами, отладчик и прочие дополнительные инструменты, упрощающие процесс разработки скриптов. Этому вопросу посвящён раздел~\ref{sec:ide-integration};
 \item модуль, встраиваемый в основное (расширяемое) приложение.
\end{itemize}

Такое глобальное разделение системы на две части даёт следующие преимущества:
\begin{enumerate}
 \item независимость компонентов. Если возникнет необходимость заменить используемую среду разработки на другую, этого можно будет достичь, минимально меняя второй компонент системы;
 \item стабильность. В случае аварийного завершения (или каких-либо ошибок) внутри среды разработки работа основного приложения с большей долей вероятности не будет затронута;
 \item безопасность. Доступ к данным основного приложения строго определяется протоколом взаимодействия компонентов системы.
\end{enumerate}

Описанные компоненты системы выполняются в разных процессах. Это добавляет им независимости. Помимо этого, забегая вперёд, отметим, что это позволит проще реализовать отладку. Таким образом, нужно выбрать способ межпроцессного взаимодействия, который позволит максимально просто и эффективно взаимодействовать компонентам системы. Этому вопросу посвящен раздел~\ref{sec:ipc}.
\subsection{Интеграция со средой разработки}

Один из важнейших сценариев использования расширяемых и автоматизируемых программных продуктов --- возможность разработки новых скриптов и редактирование кода старых в интегрированной среде разработки.

Встроенная среда разработки как правило поставляется вместе с продуктами для автоматизации приложений, например VBA или VSTA. В платформах, основная цель которых --- поддержка плагинов, напротив, встроенных средств для редактирования кода не предоставляется. Это понятно, так как считается, что конечный пользователь не будет заниматься такой сложной задачей, как написание расширения к своему приложению. Для разработчиков же существуют отдельные пакеты (SDK) для разработки расширений в различных средах разработки.

При реализации платформы для автоматизации и расширения необходимо было предоставить удобные средства редактирования кода конечному пользователю. Эта задача может быть решена несколькими способами:

Реализация собственного редактора исходного кода расширений.

Это решение имеет как и преимущества, так и целый ряд существенных недостатков. Положительным моментом этого подхода является простота интеграции в разрабатываемую платфому. Однако, реализация действительно удобного набора инструментов, таких как подсветка синтаксиса, автодополнение, автогенерация кода по шаблонам, управления файлами с исходным кодом, будет очень сложной задачей. Кроме того, результат в любом случае наврядли сравниться с современными IDE.

Использование одной из существующих IDE.

Это решение хорошо тем, что мы получаем готовую среду разработки с множеством встроенных инструментов. Одноко, встает ряд вопросов, которые необходимо решить для успешного применения сторонней IDE в разрабатываемой платформе.

Интеграция в платформу. Под интеграцией необходимо понимать возможность программного управления сторонней IDE на уровне, достаточном для того, чтобы у пользователя складывалось ощущение, что он работает с единой системой, а не несколькими разными программами.

Большинство IDE обладают куда более широким, чем нужно для разработки расширений, функционалом. Поэтому важным критерием выбора сторонней IDE является ее настраиваемость и конфигурируемость. Это касается как доступных пользователю функций, так и внешного вида самой среды.

Несмотря на предыдущий пункт, в котором сказано, что функционал современных сред разработки излишен для использования их в качастве редактора расширений, может случиться так, что некоторых специфичных функций будет нехватать. Поэтому искомая среда должна также обладать возможностью расширения функционала. Отдельным плюсом будет поддержка плагинов. Тогда можно будет отключать ненужные плагины и разработать плагины, добавляющие новый функционал.

\TODO{обзор средств}

\subsubsection{Вывод}

В качестве внешней IDE для разработки расширений была выбрана IDE с открытым исходным кодом SharpDevelop. Этот выбор был сделан по нескольким причинам:

Во-первых, эта IDE может управляться программно извне при помощи встроенной технологии SDA. Программное управление позволяет выполнять различные действия в IDE без участия пользователя. Например, показывать или скрывать само приложение, запускать сборку проекта, управлять сохранением и загрузкой проектов, и многие другие. 

Во-вторых, эта IDE поддерживает плагины. Это свойство позволит добавить какую-либо отсутствующую в базовой поставке функциональность не меняя исходного кода самого SharpDevelop. Это важно, так как изменение исходного кода сделает неудобным распространение готовой платформы, так как она будет совместима только с конкретной сборкой SharpDevelop. Использование плагинов позволит добавить нужную функцинальность в уже установленный экземпляр этой IDE.

В-третьих, SharpDevelop поддерживает кастомизацию, то есть изменение своего внешного вида и поведения при момощи конфигурационных файлов. Под изменением внешнего вида нужно понимать удаление <<лишних>> в контексте данного сценария использования (как внешней IDE для разработки расширений) элементов управления, добавление новых элементов управления, управление доступностью этих элементов, и прочее. Изменение поведения состоит в подмене штатных обработчиков событий элементов управления (например, кнопок), на свои собственные обработчики с измененной логикой. Например, запуск отладчика не должен пытаться стартовать сборку самого расширения (это попросту невозможно, так как расширения представляет из себя библиотеку классов), а присоединить отладчик к процессу хост-приложения.

В-четвертых, открытый исходных код поможет быстрее решить проблемы, связанные с интеграцией SharpDevelop в разрабатываемую платформу. Код IDE может быть отредактирован благодаря открытой лиценции GNU GPL v2. IDE может быть запущена под отладчиком для выявления и локализации проблем вызванных внешними факторами и нетипичным использованием самой IDE.

\pagebreak

\subsection{Межпроцессное взаимодействие}
\label{sec:ipc}

В каждой операционной системе предоставляются различные способы межпроцессного взаимодействия. Ниже перечислены наиболее популярные из них:
\begin{itemize}
   \item использование доступа к общему файлу;
   \item использование сигналов;
   \item сокеты;
   \item именованные каналы;
   \item разделяемая память;
   \item очереди сообщений.
\end{itemize}

Некоторые способы можно реализовывать вручную, для некоторых есть специальные функции в API операционной системы. Однако более высокоуровневые библиотеки и платформы, такие как {\it .NET}, предоставляют обобщённые и более универсальные механизмы для межпроцессного взаимодействия. Ниже будет рассмотрен один из них.

{\it WCF (Windows Communication Foundation)} --- программный фреймворк, используемый для обмена данными между приложениями, входящими в состав {\it .NET Framework}. {\it WCF} делает возможным построение безопасных и надёжных транзакционных систем через упрощённую унифицированную программную модель межплатформенного взаимодействия. Если не вдаваться в подробности, можно сказать, что {\it WCF} --- некая абстракция, обёртка, которая скрывает фактические детали реализации межпроцессного взаимодействия, которое на низком уровне будет реализовано за счёт одного из вышеперечисленных способов. 

Для осуществления межпроцессного должен быть написан интерфейс, который будет реализован в одном из процессов. Процесс, реализующий интерфейс, предоставляет сервис (службу), к которой может подключиться другой процесс, удалённо вызывая методы интерфейса. Для обратного взаимодействия можно использовать callback-вызовы, либо создавать и реализовывать ещё один интерфейс и открывать второй канал для взаимодействия в обратную сторону. 

В разрабатываемой платформе межпроцессное взаимодействие нужно для:
\begin{itemize}
 \item управления средой разработки;
 \item передачи дополнительных данных среде разработки;
 \item передача событий среды основному приложению.
\end{itemize}


\pagebreak

\subsection{Интеграция расширений}
\label{sec:extention_interaction}

\TODO{обзор возможных технологий, выводы}

\pagebreak

\subsection{Интеграция в расширяемое приложение}

\TODO{обзор возможных технологий, выводы}

\pagebreak

\subsubsection{Архитектура Системы}

\begin{figure}[!h]
    \centering
    \includegraphics[width=15cm]{fw_arch1.jpg}
    \caption{Взаимодействие компонентов системы}
    \label{fw_arch1}
\end{figure}


\pagebreak


\subsubsection{Архитектура Системы}

\begin{figure}[!h]
    \centering
    \includegraphics[width=15cm]{fw_arch1.jpg}
    \caption{Взаимодействие компонентов системы}
    \label{fw_arch1}
\end{figure}


\subsection{Применение разработанной платформы на существующем Open-Source проекте}

\TODO{Да, я искренне в это верю!}

\pagebreak

	\section{Сравнение с аналогами}

Единственным непосредственным аналогом разработанной платформы является VSTA. (речь идет о ПО для платформы .NET)  Сравнение будет происходить основываясь на опыте применения технологии VSTA в реальном проекте (см раздел \ref{sec:use_exis_techn}). Кроме того, в рамках данной работы необходимо оценить успешнось разработанной платформы, как замены устаревшего инструментария VBA, так как один из сценариев применения платформы - замена VBA при портировании COM/ActveX-приложений, использующих VBA, на платформу .NET.

\subsubsection{Методика сравнительного анализа}

Рассматриваемые плаформы поддержки расширений можно представить как совокупность следующих компонент:

\begin{itemize}
   \item среда разработки расширений;
   \item пользовательский интерфайс (инструментарий для работы с расширениями);
   \item ядро, обеспечивающее взаимодействие приложения с расширениями.
\end{itemize}

Для получения адекватных результатов сравнительный анализ необходимо проводить по каждой из этих компонент. Так как сформулировать критерии для получения точных количественных оценок при сравнении, к примеру, удобства пользовательского интерфейса, не представляется возможным, все выводы будут следовать по большей части личного опыта использования тех или иных продуктов и инструментов, а так же из отзывов других разработчиков или пользователей этих продуктов.

Кроме перечисленных компонент системы немаловажным фактором в выборе того или иного продукта будет являться простота его использования. В конце раздела будет проведено сравнение сложности интеграции продуктов в приложение.

\subsubsection{Среда разработки}

VSTA Studio является логичным развитием VB6 Studio, используемой как редактор VBA-кода. Большинство основных элементов управления сргуппированы аналогично и выполняют аналогичные операции в этих IDE. Однако, VSTA имеет более <<продвинутый>> визуальный редактор форм, а так же новые инструменты IntelliSense, увеличивающие эффективность разработки. Используемая в разработанной платформе IDE SharpDevelop имеет похожий функционал, однако набор возможностей редактора кода не столь широк, как в VSTA Studio. В этом мы убедились ранее, реализовав генератор сигнатур обработчиков событий, отсутствующий в SharpDevelop по умолчанию. (см. раздел \ref{sec:ehsg}) Так же, из-за того что использование этой IDE не подразумевает сценария использования как интегрированой среды разработки расширений, проявилось множество проблем, требующих разрешения. (поднобнее про эти проблемы и методы их решения см. раздел \ref{sec:dev-details}) Не смотря на отсутствие некоторых функций в базовом установочном пакете SharpDevelop, его функционал может быть расширен засчет механизма плагинов, либо интегрированной технологии SDA.

Средства разработки компании Microsoft предоставляют более шировий набор возможностей <<из коробки>>, однако используемая в разработанной платформе IDE является более гибким и масштабируемым решением. Кроме того, SharpDevelop является полноценной IDE для разработки программ на большом числе .NET-совместимых языков программирования (C\#, VisualBasic.NET, F\#, Boo, IronPython, IronRuby и другие, для которых существуют плагины SharpDevelop, поддерживающие их), в то время как продукты Microsoft поддерживают только Visual Basic и имеют менее богатый встроенный инструментарий для разработчика.

\subsubsection{Пользовательский интерфейс}

Как таковые, графические средства для проведения операций с расширенями продукты Microsoft не предоставляют. То есть разработчику программного обеспечения необходимо реализовать свой собственный инструментарий в соответствии с поставлеными целями. В разработанную платформу интегрирован пользовательский интерфейс, реализованный на Windows Forms, предоставляющий основные инструменты для управления расширениями. (см. раздел \ref{sec:macro-gui})

Преимущества такого решения очевидны: разработчику программного обеспечения достаточно встроить готовые компоненты в свое приложение чтобы получить полнофункциональную систему боддержки расширений. Из недостатков стоит отметить неприемлемость такого решения в случае, если втроенные в платформу элементы нарушают целостность визуального оформления приложения. В этом случае имеет смысл использовать один из паттернов проектирования для предоставления разработчику возможность самому реализовать представления компонент графического интерфейса в соответствии со стилем разрабатываемого приложения.

\subsubsection{Взаимодействие расширения и приложения}

В этом аспекте не имеет смысла сравнивать разработанную платформу с VBA из-за принципиального отличия модели взаимодействия COM-компонент и .NET-библиотек между собой. В свою очередь, используя VSTA можно выбрать различные варианты работы с ним. В одном из случаев взаимодействие расширения и приложения будет происходить через System.Addin, особенности этого подхода были подробно описаны в разделе \ref{sec:system_addin}. В разрабатываемой платформе был выбран .NET Reflection в качестве основы для организации взаимодействия. По большому счету это и евляется основным отличием разработанной платформы от VSTA.

Эта часть платформы хоть и играет, пожалуй, самую важную роль, но скрыта от пользовательских глаз. Поэтому, если все пользовательские сценарии реализованы верно и не возникает проблем со стабильностью и производительностью, по большому счету не имеет смысла утверждать что тот или иной подход оказывается лучше или хуже. Минусом исполизуемого подхода (см. раздел \ref{sec:extention_interaction}) является тот факт, что сборки расширения остаются в домене приложения, то есть в памяти, до завершения работы с ним. Это сложно назвать серьезным недостатком, так как эти <<паразитные>> сборки полностью изолируются и не могут повлиять на работу приложения. Более того, их размер ничтожно мал по сравнению с доступными объемами памяти, иварьируется от десятков до сотен килобайт, в зависимости от объема кода расширения. В исключительных случаях, если расширение перегружено разнообразными ресурсами, размер может доходить до нескольких мегабайт. В конце-концов, так как проблема <<паразитных>> сборок имеет место только в случае активного использования отладчика расширения и редактирования его кода, она не будет мешать работе с уже готовыми расширениями.

\subsubsection{Интеграция}

Основным преимуществом разработаного решения является относительная простота его интеграции как в существующее приложение, так и на этапе разработки нового приложения. Простота достигается благодаря использованию разработанного механизма взаимодействия расширения и приложения, в основе которого лежит .NET Reflection. Подробнее про особенности реализованного механизма написано в разделе \ref{sec:extention_interaction}. Для предоставления доступа к объектам программисту требуется всего лишь реализовать этими объектами интерфейс, используемый модулем интеграции расширений разработанной платформы. В то же время, интеграция рассмотренных в обзоре (см. раздел \ref{sec:extention_interaction}) требует реализации сложных и громоздких протоколов взаимодействия на основе контрактов, а так же накладывает некоторые условия на архитектуру разрабатываемого приложения. Конечно, таким образом достигается изоляция расширения от приложения, но при этом интеграция сильно усложнена.

Так же стоит отметить наличие в разработанной платформе средств для управления расширениями, которые могут быть легко добавлены полностью или частично в целевое приложение на этапе интеграции. Если же такой вариант разработчика не устроит, он может реализовать собственные визуальные компоненты, однако, их придется интегрировать в платформу поддержки расширений, на что будет потрачено дополнительное время.

\subsubsection{Выводы}

Исходя из проведенного выше анализа следует, что разработанная платформа имеет ряд преимуществ над большинством существующих решений:

\begin{itemize}
   \item Наличие полноценной среды разработки расширений;
   \item Более простая интерация в существующие приложения;
   \item Возможность отладки расширений;
\end{itemize}

\pagebreak


\section{Перспективы развития платформы}

Несмотря на то, что основные требования к платформе были удовлетворены, существует возможность реализовать дополнительные инструменты и расширить функциональность платформы для обеспечения большего удобства и достижения максимальной эффективности ее использования. Так же, реализация некоторых дополнительных функций может способствовать расширению потенциальной аудитории пользователей данной платформы. На взгляд автора, у данной разработки существует несколько возможных направлений развития:

\begin{itemize}
   \item Развитие и реализация инструментов для упрощения и частичной автоматизации интеграции платформы;
   \item Реализация дополнительных инструментов для обеспечения эффективности и упрощения процесса разработки расширений на базе данной платформы;
   \item Улучшение механизмов управления расширениями; распространения расширений;
   \item Своевременная реализация поддержки новых версий SharpDevelop и .NET Framevork.
\end{itemize}

Рассмотрим некоторые возможности развития платфоры более подробно.

\subsubsection{Инструменты интеграции}

Для упрощения задачи интеграции платформы в готовое приложение (а именно с такой задачей ) имеет смысл продумать и реализовать набор инструментов для анализа кода приложения, выделения объектов, к которым хочется предоставить доступ расширению и автоматической генерации кода, реализующего нужные ядру платформы интерфейсы взаимодействия расширение-приложение для этих объектов. Эти инструменты так же будут полезны и при реализации нового приложения, использующего представленную в работе платформу, так как позволят автоматизировать процесс.

\subsubsection{Инструменты управления расширениями}

Так как один из основных сценариев использования платформы --- редактирование кода расширения (правка ошибок, реализация новой фанкциональности, и т. д.), имеет смысл создать инструменты для коллективной разработки расширений, например реализовать возможность подключения к репозиторию, а так же возможность отправки отчета об ошибке в расширении (возможно, содержащего и исправление этой самой ошибки) разработчику.

Так же не лишней будет проработка системы распространения расширений для каждого конкретного приложения, на подобие популярных ныне AppStore и Play market.

\subsubsection{Обеспечение эффективности разработки}

Возможно возникновение ранее не предусмотренных сценариев использования SharpDevelop, когда его встроенных возможностей будет не хватать для обеспечения максимальной эффективности процесса разработки расширения. В этом случае может понадобиться реализация плагинов и/или других модулей для добавления интересующей функциональности, как это было сделано в разделе ~\ref{sec:ehsg}).

\subsubsection{Взгляд в будущее}

Помимо рассмотренных выше перспектив, можно предположить теоретически возможные для данного продукта сценарии использования. Например, стоит рассмотреть возможность понижения <<порога вхождения>> для конечного пользователя. Конечно, для реализации простых макросов пользователю не требуются глубокие знания программирования и владение множеством технологий и инструментов. Однако, базовые знания конструкций языка и возможностей самой платформы все еще остаются необходимы. Более того, каждое приложение с поддержкой пользовательских макросов требует некоторого время на обучение пользователя работе с необходимым  инструментарием. Этот фактор являются сдерживающим для многих начинающих пользователей работа которых могла бы стать куда более эффективной с использованием возможностей автоматизации. Решением этой проблемы может стать <<визуальное программирование>>. Уже давольно давно появились и с успехом применяются средства графического программирования, такие как {\it JMCAD}, {\it LabVIEW}, {\it HiAsm} и другие.

Отдельного внимания заслуживает программа {\it Automator}, разработанная компанией Apple. Она позволет по принципу drag-and-drop создавать скрипты для автоматического выполнения различных действий. В Automator можно использовать большое количество готовых программных блоков, выполняющих действия с использованием таких программ, как Finder, Safari, iCal, Address Book и т. д. Он позволяет также использовать и сторонние программы. Несмотря на то что Automator использует AppleScript и/или Cocoa, для использования программы не требуется знаний этих языков, все действия выполняются полностью в графической среде, хотя существуют блоки для вставки, позволяющие выполнить код на AppleScript или в shell-среде. Общий принцип действия достаточно прост: выходные параметры одного действия являются входными параметрами следующего, действия выполняются поочерёдно, при этом также имеются способы как повторов действий, так и их зацикливания.~\cite{automator-website} Принципы, лежащие в основе программы {\it Automator} можно применить для создания так называемого <<конструктора макросов>>, который, являясь частью разрабатываемой платформы может сильно повлиять на рынок средств автоматизации программных продуктов.

Таким образом, данную работу так же можно рассматривать как некоторый <<задел>> для будущих разработок в этом направлении.

\pagebreak


\setcounter{secnumdepth}{0}
\section{Заключение}
\setcounter{secnumdepth}{2}

В данной работе затронута одна из проблем, возникающих при разработке современных программных комплексов: проблема автоматизации и расширения ПО. Большинство современных крупных приложений так или иначе поддерживают автоматизацию либо расширение. В некоторых ситуациях для решения этой задачи используются какие-нибудь готовые разработки, в некоторых случаях реализуется какой-то специфичный механизм. Существует несколько готовых программных решений для интеграции возможностей автоматизации и расширения в приложения. В работе рассмотрены наиболее популярные из них. 

Необходимость интеграции возможностей автоматизации и расширения в приложение возникла на реальном коммерческом проекте. В рамках проекта необходимо было портировать приложение, разработанное с использованием устаревших технологий, на современные платформы. Автоматизация и расширение достигались в приложении за счёт технологии {\it Visual Basic for Applications}, которая, хоть и является на настоящий момент устаревшей, весьма успешно решала поставленную задачу. Найти подходящую замену для {\tt .NET} оказалось непросто. Наилучшим кандидатом казалась платформа {\it Visual Studio Tools for Applications}, которая и была внедрена в разрабатываемое приложение. Однако как на этапе разработки, так и не этапе тестирования возникло множество проблем. Более того, на момент окончания разработки выяснилось, что {\it VSTA} больше не поддерживается и лицензию на неё приобрести невозможно.

В результате было принято решение разрабатывать новую платформу, позволяющую интегрировать возможности расширения и автоматизации в приложения. При разработке платформы учитывались результаты исследования существующих решений, а также опыт внедрения одного из них на реальном проекте, разрабатываемом в компании First Line Software.

Разработанная платформа отвечает всем поставленным требованиям. Она нацелена на упрощение процесса интеграции возможностей автоматизации и расширения в приложения. Помимо модулей, встраиваемых в приложения, платформа содержит ряд утилит для анализа и генерации кода, что ускоряет процесс интеграции. Созданная платформа была внедрена в проект компании First Line Software. Результаты работы были высоко оценены заказчиками.

Разработанная платформа сравнивалась по основным критериям с существующими решениями, и в результате анализа был сделан вывод о том, что она действительно имеет ряд преимуществ перед существующими разработками.

\pagebreak
	% -------------------------------------------------------------
    
	
	% ------------  for testing purposes ----------------------
	% Здесь указаны источники, на которые ПОКА НЕТ ССЫЛОК ИЗ ТЕКСТА. Всего источников: 29
	%\cite{band-four}
	%\cite{plux-website}
	%\cite{mef-website}
	%\cite{alplatform-website}
	%\cite{mono-addins-website}
	%\cite{announcing-sda-article}
	%\cite{vsto-website}
	%\cite{vsta-website}
	%\cite{System.Addins-article}
	%\cite{addins1-article}
	%\cite{addins2-article}
	%\cite{automata-via-macros}
	%\cite{mastering-vba}
	%\cite{addins-and-extensibility}
	%\cite{dotnet-app-extensibility}
	%\cite{use-systemaddin-namespace}
	%\cite{wcf-services}
	%\cite{wcf-unleashed}
	
	%\cite{patterns-of-enterprise-app}
	%\cite{maf}
	%\cite{sharpdevelop}
	%\cite{writing-sd-addin}
	%\cite{use-sd-core}
	%\cite{vs-website}
	%\cite{monodevelop-website}
	%\cite{CLR-via-CS}
	%\cite{cs2010-dotnet40}
	%\cite{cs2008-dotnet35}
	%\cite{automator-website}
	
	% -------------------------------------------------------------
	
    
    % Библиография.
    % Как описано здесь: http://en.wikibooks.org/wiki/LaTeX/Bibliography_Management
    % TODO :: Установить требуемый Правилами стиль библиографии.
    \bibliographystyle{unsrt} % в порядке упоминания в тексте
    \cleardoublepage                           % !!!!! TODO :: такое решение сработает вроде как не всегда, иногда будет пустая страница !!!!!
    \addcontentsline{toc}{section}{Источники} % !!!!! TODO :: это баг!!!!!!! литература начнётся с новой страницы !!!!!!!!!!!!!!!!!!!!!!!!!!!
    \bibliography{references}
    
    %\appendix
    
	
\end{document}
