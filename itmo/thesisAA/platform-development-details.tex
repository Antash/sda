\section{Разработка платформы}
\label{sec:dev-details}

\subsubsection{Архитектура Системы}

\begin{figure}[!h]
    \centering
    \includegraphics[width=15cm]{fw_arch1.jpg}
    \caption{Взаимодействие компонентов системы}
    \label{fw_arch1}
\end{figure}


\subsection{Кастомизация SharpDevelop}
\label{sec:sd_custom}

Среда разработки SharpDevelop имеет встроенный механизм поддержки плагинов~\cite{sharpdevelop}. Более того, ядро IDE представляет из себя всего лишь платформу для поддержки плагинов~\cite{use-sd-core}, частью которой и является SDA, рассмотренная ранее. Функционал SharpDevelop реализуется при помощи целой иерархии взаимосвязанных плагинов, хранящихся в папке {\it \\AddIns}. Каждый плагин имеет свой собственный xml-файл конфигурации {\it *.addin}, в котором размещается информация, необходимая ядру для интеграции плагина~\cite{writing-sd-addin}. 

Помимо отдельных конфигурационных файлов для каждого плагина, существует отдельно лежащий конфигурационный файл {\it ICSharpCode.SharpDevelop.addin}. Этот файл определяет поведение сборки {\it ICSharpCode.SharpDevelop.dll}, которая сама по себе также является плагином, но очень крупным. Можно сказать, что эта сборка и реализует базовый функционал IDE.

В файле содержится несколько секций, рассмотрим подробно те, которые будут использоваться для кастомизации: 

\begin{itemize}
 \item <AddIn> Заголовочная секция, содержит имя сборки, описание, автора, а также определяет видимость в менеджере плагинов(видимый через менеджер плагин можно отключить)
 \item <Manifest> Хранит имя и версию сборки
 \item <Runtime> <Import> Определяет сборки, загружаемые ядром.
 \item <Path> Содержит путь к визуальному элементу управления типа <<контейнер>>. Внутри этой секции располагается описание различных элементов, которые находятся в контейнере, а также их поведение.
   \subitem <MenuItem>, <ToolbarItem> Элементы управления. Могут содержать информацию о связаных ресурсах (иконка, текст, всплывающая подсказка), а также имя класса-обработчика.
   \subitem <Condition>, <ComplexCondition> Условие, либо составное условие, описывающее будет ли отображаться тот или иной элемент управления в определенных ситуациях.
\end{itemize}

Таким образом при помощи конфигурационных файлов можно реализовать несколько сценариев кастомизации:

\begin{itemize}
 \item Загрузка дополнительных сборок в адресное пространство SharpDevelop
  \subitem с целью подменять обработчики событий существующих элементов управления на свои собственные;
  \subitem с целью создания своих собственных элементов управления различного типа (кнопки, панели инструментов, пункты меню, и т. д.)
 \item Комментирование кода в конфигурационном файле для удаления ненужных элементов управления или выключения загрузки сборок.
 \item Добавление или изменение условий.
\end{itemize}
 
Помимо перечисленных сценариев кастомизации SharpDevelop и его компонент с использованием конфигурационных файлов, есть возможность реализации плагинов для этой среды разработки. В рамках этой работы создание плагинов рассматриваться не будет, так как все необходимые действия удалось совершить при помощи конфигурационного файла.
 
Для решения задачи кастомизации была создана библиотека, содержащая обработчики следующих событий SharpDevelop (в скобках указано новое действие):

\begin{itemize}
 \item Создание нового проекта (вызывает соответствующий диалог создания расширения);
 \item Открытие существующего проекта (диалог открытия проекта расширения из файла/базы данных/архива);
 \item Сохранение проекта/файла (диалог сохранения проекта в файл/базу данных/архив);
 \item Сборка проекта (в случае успешной сборки, обработчик готовит исполняемый файл расширения к загрузке в адресное пространство приложения);
 \item Старт отладки (загружает файл расширения в адресное пространство приложения и присоединяет к его процессу встроенный в SharpDevelop отладчик);
\end{itemize}

Эти события требуют интеграции в хост-процесс IDE для обеспечения взаимодействия с расширяемым приложением. Реализация своих собственных обработчиков требовала детального изучения исходного кода и архитектуры SharpDevelop, для обеспечения стабильности такого решения. Пример исходного кода обработчика с комментариями можно найти в приложении 2; фрагменты измененного файла конфигурации {\it ICSharpCode.SharpDevelop.addin} см. приложение 1.
 
 %-----------------------------------------------------------------
 
\subsection{Управление сборками расширений}
\label{sec:dll_manip}

Один из сценариев использования платформы --- редактирование кода расширения. При этом подразумевается, что отредактированный проект расширения будет перекомпилирован и обновлённый бинарный файл будет загружен в адресное пространство расширяемого приложения. При этом возникает несколько проблем:

\begin{itemize}
  \item необходимо также загружать {\it .pdb} файлы, содержащие информацию для отладчика;
  \item после загрузки библиотеки приложением она более не может быть выгружена из домена приложения, соответственно файл библиотеки блокируется;
  \item необходимо, чтобы компилятор, вызываемый из среды {\it SharpDevelop}, всегда мог перезаписать библиотеку расширения после её компиляции.
\end{itemize}

Теоретически, для организации работы с файлами библиотек может быть использован один из нескольких подходов:

\begin{enumerate}
  \item Загрузка бинарного файла и отладочной информации в домен приложения в виде потока байт {\it MemoryStream}.
  При этом подходе файл библиотеки не блокируется приложением и может быть повторно перезаписан при последующих сборках расширения. Однако при попытке использования такого подхода оказалось, что отладчик SharpDevelop не умеет подгружать отладочную информацию из потока байт. От этого способа пришлось отказаться.
  \item Копирование бинарного файла и отладочной информации в специально отведенное для этого хранилище, из которого они, в свою очередь, будут загружены в приложение.
  Существует как минимум два варианта реализации этого подхода.
  \begin{enumerate}
    \item Хранение всех версий сборок в одной папке в виде файлов с уникальным случайным именем. Такой способ также не принес результатов, так как и в этом случае отладчик SharpDevelop отказался находить отладочную информацию.
	\item Создание для каждой версии сборки своей временной папки. Наконец, в этом случае с отладчиком не возникло проблем, и было решено использовать именно такой сценарий хранения файлов расширений.
	\end{enumerate}
\end{enumerate}

Для реализации задачи хранения сборок было решено использовать изолированное хранилище~\cite{cs2010-dotnet40}. Оно имеет определённые преимущества по сравнению с другими возможными способами организации хранения. Во-первых, изолированное хранилище доступно даже если у пользователя нету привилегий в файловой системе. Во-вторых, оно достаточно хорошо скрыто от посторонних глаз в профиле пользователя. Наконец, для реализации работы с изолированным хранилищем в .NET существует объект {\it IsolatedStorageFile}. Он служит для создания различных файлов и папок в изолированном хранилище и проведения манипуляций с ними. После окончания работы с программой, изолированное файлы в хранилище могут быть удалены, или же оставлены. При этом гарантирована их доступность при повторном запуске приложения. В данном случае временные сборки не представляют интерес, и будут удаляться по завершению работы с приложением.

 %-----------------------------------------------------------------
 
\subsection{Интеграция с отладчиком}
\label{sec:sd_debug}

В процессе тестирования отладчика скриптов, стало понятно, что работать с ним не очень удобно. Основная причина этого заключалась в том, что при остановке выполнения скрипта на точке останова поток приложения, в котором выполняется этот скрипт блокируется, но при этом больше никаких действий не происходит, в худшем случае пользователь видит <<зависшее>> приложение. То есть пользователю каждый раз, когда приложение <<зависает>>, приходится переключаться на окно IDE и проверять, не остановилось ли выполнение скрипта отладчиком. Такое поведение недопустимо. Необходимо, чтобы при остановке на брейкпоинте окно SharpDevelop сообщало об этом, например, его статус бы менялся на <<Поверх всех окон>>.

Казалось бы, для реализации подобного поведения необходимо изменить код SharpDevelop, добавив в событие точки останова <<BreakpointHit>> код для изменения статуса главного окна. Однако, изменение исходного кода повлекло бы за собой проблемы при распространении платформы, которая не могла бы использовать существующий инсталляционный пакет SharpDevelop, а требовала установки модифицированной его версии. Это также неудобно, если пользователь собирается использовать SharpDevelop для других целей. Такое изменения поведения в этом случае нежелательно.

Задача изменения реакции на внутреннее событие была решена при помощи рассмотренного ранее механизма {\it .NET Reflection}~\cite{CLR-via-CS}. Ниже приведены фрагменты кода класса {\it SDBreakpointService} с комментариями.

Объявление дополнительного обработчика события достижения точки останова:

\begin{figure}[!h]
    \includegraphics[width=16cm]{code1.png}
\end{figure}

%\begin{lstlisting}
%var bpDebuggerBreakpointHit 
%     = new EventHandler<Debugger.BreakpointEventArgs>(
%        delegate
%        {
%              SDIntegration.Instance.BringToFrontIDE();
%        }
%);
%\end{lstlisting}


Получение коллекции точек останова, установленных в открытом проекте:	

\begin{figure}[!h]
    \includegraphics[width=16cm]{code2.png}
\end{figure}

%\begin{lstlisting}
%var fieldInfo = DebuggerService.CurrentDebugger.GetType().GetField(
%      "debugger", BindingFlags.Instance | BindingFlags.NonPublic);
%var debugger = fieldInfo.GetValue(DebuggerService.CurrentDebugger);
%var field = debugger.GetType().GetField("breakpointCollection", 
%     BindingFlags.Instance | BindingFlags.NonPublic);
%var breakPointColl = (IEnumerable)field.GetValue(debugger);
%\end{lstlisting}

Как видно, здесь происходит обращение к закрытым членам класса {\it Debugger} из плагина {\it Debugger.Core.dll}. Это нужно учитывать при миграции платформы на следующие версии SharpDevelop, так как логика работы этого класса может измениться.

Далее, всем точкам останова присваивается дополнительный обработчик события:

\begin{figure}[!h]
    \includegraphics[width=16cm]{code3.png}
\end{figure}

%foreach (var bp in breakPointColl)
%{
%	bp.GetType().GetEvent("Hit").AddEventHandler(bp, bpDebuggerBreakpointHit);
%}

Стоит отметить, что эти действия необходимо повторять для каждого нового брейкпоинта, установленного в IDE. К счастью, событие установки брейкпоинта существует, и доступно <<из коробки>>, поэтому дополнительных проблем с этим не возникло.

Также стоит отметить, что в SharpDevelop начиная с версии 4.1 наконец-то появилась опция автоматической остановки выполнения приложения с подключенным отладчиком в случае возникновения исключительной ситуации. Благодаря этой функции разработчику расширения будет понятно, в каком месте кода произожло исключение. Напомню, в ранних версиях SharpDevelop (4.0 и ниже) такой возможности не было, и в случае необработанного исключения приложение просто завершилось бы с ошибкой. В такой ситуации исправлять ошибки в коде расширения довольно затруднительно. 

 %-----------------------------------------------------------------

\subsection{Реализация пользовательского интерфейса управления расширениями}
\label{sec:macro-gui}

Одним из требований к разрабатываемой платформе являлась реализация средств управления расширениями. Для этого требовалось определение пользовательских сценариев работы с системой.

Пользователь системы должен иметь возможность:

\begin{itemize}
  \item создавать новое расширение;
  \item открывать среду разработки расширения;
  \item сохранять расширение (в файл архива, папку или базу данных, в зависимости от желания разработчика приложения);
  \item загружать расширение из вышеупомянутых источников.
\end{itemize}

В целом, реализация элементов интерфейса не содержит интересных вопросов, требующих детального рассмотрения в рамках работы. На рисунке \ref{pic:mmui} можно увидеть скриншот главного экрана приложения, в которое была интегрирована платформа, и открытое окно загрузки проекта расширения.

Отдельно стоит упомянуть, что реализованные графические элементы управления также должны быть доступны из среды разработки. То есть при нажатии кнопок создания, загрузки или сохранения в среде {\it SharpDevelop} пользователь должен видеть не стандартные диалоги этой IDE, которые не несут в данном контесте никакой функциональной нагрузки, а диалоги, предоставляемые платформой. Это требование также было реализовано при помощи конфигурационных файлов {\it SharpDevelop} (см. раздел \ref{sec:sd_custom}).

\begin{figure}[!h]
    \centering
    \includegraphics[width=16cm]{mmui.png}
    \caption{Главное окно приложения, в которое интегрирована разработанная платформа. Открыто окно загрузки расширения}
    \label{pic:mmui}
\end{figure}

 %-----------------------------------------------------------------

\subsection{Организация доступа к объектам приложения}
\label{sec:ext_entry_point}

Для реализации возможности доступа к объектам расширяемого приложения в коде расширения был создан генератор дизайн-файлов, хранящих объявление списка доступных объектов. Систама реализована таким образом, чтобы постоянно поддерживать список объектов в этих дизайн-файлах в актуальном состоянии. Это необходимо для предоставления разработчику расширения актуальной информации о доступных объектах.

На рисунке \ref{pic:ext_design} можно увидеть, в каком виде представляется описанный список объектов, доступный разработчику расширения. Видно, что в показаном partial-классе сосредоточены объекты одного из типов, которых в данном приложении доступно три. Каждому типу объектов соответствует своя коллекция, которая отражается в динамически генерируемых классах-обёртках, расположенных в специально отведенной папке проекта расширения. Помимо списка объектов класс-обёртка содержит обработчики системных событий, таких как инициализация компонент приложения, начало и завершение работы, и другие.

\begin{figure}[!h]
    \centering
    \includegraphics[width=16cm]{ext-design.png}
    \caption{Один из дизайн-файлов макропроекта на языке VisualBasic.NET, открытого в SharpDevelop}
    \label{pic:ext_design}
\end{figure}
 
 Еще раз обращаю внимание на то, что каждое расширение хранит информацию о всех доступных извне объектах приложения в специальной группе дизайн-файлов, которые поддерживаются постоянно в актуальном состоянии. Эти объекты могут быть любым способом использованы в коде расширения.
 
  %-----------------------------------------------------------------

\subsection{Реализация генератора сигнатур обработчиков событий}
\label{sec:ehsg}

Генератор сигнатур --- одна из функций, отсутствующих по уполчанию в {\it SharpDevelop}, но при этом присутствующая в {\it VBA} и {\it VSTA} студиях. Так как без генератора сигнатур реализовывать обработчики событий крайне неудобно (требуется каждый раз восстанавливать сигнатуру метода-обработчика по памяти либо исходя из текста сообщений ошибок на этапе компиляции), было решено добавить недостающую функциональность в {\it SharpDevelop}.

Необходимая функциональность была добавлена в {\it SharpDevelop} при помощи редактирования конфигурационного файла (см. раздел \ref{sec:sd_custom}). Таким образом был добавлен новый элемент управления (жёлтая <<молния>> на рисунке \ref{pic:sdesg}), подписанный на обработчик, находящийся в коде модуля интеграции с IDE. Этот обработчик вызывал форму интерфейса генератора обработчиков событий (см. рисунок \ref{pic:sdesg}).

Как известно, функция-обработчик имеет делегатный тип. При помощи уже знакомой нам технологии {\it .NET Reflection} производился разбор каждого конкретного делегатного типа, при котором определялся список параметров, их имена и типы. Имея эту информацию нетрудно было сгенерировать сам код функции-обработчика в виде строки и вставить его в файл проекта при помощи технологии SDA.

На рисунке \ref{pic:sdesg} представлен пример работы с ренератором сигнатур. Он содержит два основных поля: поле выбора нужного объекта и связанный с каждым из объектов список событий, на которые может быть осуществлена подписка. После выбора нужных пунктов в этих двух списках в окне появляется сигнатура обработчика, которая сразу же может быть добавлена в конец класса, объявленого в активном файле кода, открытом в среде разработки.

\begin{figure}[!h]
    \centering
    \includegraphics[width=16cm]{sdesg.png}
    \caption{Пример работы с генератором обработчиков сигнатур}
    \label{pic:sdesg}
\end{figure}

 %-----------------------------------------------------------------
 
\subsection{Улучшение общей отказоустойчивости системы}
\label{sec:stability}

В процессе тестирования платформы на реальном проекте была отмечена довольно низкая общая надежность связки <<Приложение --- хост-процесс IDE --- SharpDevelop>>. Так как бесперебойная работа системы зависит от корректного выполнения сразу двух процессов, а также безошибочного обмена данными между ними, общая отказоустойчивость понижается, так как отказ в любом месте системы приводит к ее полному краху. Логично, что чем больше зависимых компонент в системе, тем менее стабильно она работает. Решить проблему можно либо повышая стабильность каждого из узлов системы, для снижения вероятности отказа, либо реализации возможности корректного восстановления работы компонента в случае его отказа. Хорошим решением является применение обоих вариантов одновременно.

Чтобы повысить общую отказоустойчивость было решено внедрить механизм автоматического восстановления корректной работы хост-процесса IDE и канала связи между процессами в нештатных ситуациях.

В случае ошибки в хост-процессе, он <<сообщает>> об этом приложению и перезапускается. При этом обмен данными приостанавливается до полного восстановления работоспособности каналов обмена. При обрыве канала также происходит его переинициализация.

\pagebreak