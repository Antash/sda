Для написания пользовательских расширений могут использоваться как скрипты (в терминологии некоторых программ «макросы»), так и плагины (независимые модули, написанные на компилируемых языках).

\subsubsection{Скриптовый язык удобен в следующих случаях:}

\begin{enumerate}
\item Если нужно обеспечить программируемость без риска дестабилизировать систему. Так как, в отличие от плагинов, скрипты интерпретируются, а не компилируются, неправильно написанный скрипт выведет диагностическое сообщение, а не приведёт к системному краху;
\item Если важен выразительный код. Во-первых, чем сложнее система, тем больше кода приходится писать «потому, что это нужно. Во-вторых, в скриптовом языке может быть совсем другая концепция программирования, чем в основной программе — например, игра может быть монолитным однопоточным приложением, в то время как управляющие персонажами скрипты выполняются параллельно. В-третьих, скриптовый язык имеет собственный проблемно-ориентированный набор команд, и одна строка скрипта может делать то же, что несколько десятков строк на традиционном языке. Как следствие, на скриптовом языке может писать программист очень низкой квалификации — например, дизайнер своими руками, не полагаясь на программистов, может корректировать правила игры;
\item Если требуется кроссплатформенность. Хорошим примером является JavaScript — его исполняют браузеры под самыми разными ОС.
\end{enumerate}

\subsubsection{У плагинов же есть три важных преимущества:}

\begin{enumerate}
\item Готовые программы, оттранслированные в машинный код, выполняются значительно быстрее скриптов, которые интерпретируются из исходного кода динамически при каждом исполнении. Поэтому скриптовые языки не применяются для написания программ, требующих оптимальности и быстроты исполнения. Но из-за простоты они часто применяются для написания небольших, одноразовых («проблемных») программ;
\item Полный доступ к любому аппаратному обеспечению или ресурсу ОС (в скриптовом языке для этого должен существовать написанный на машинном коде API). Плагины, работающие с аппаратным обеспечением, традиционно называют драйверами;
\item Если предполагается интенсивный обмен данными между основной программой и пользовательским расширением, для плагина его обеспечить проще.
\end{enumerate}

Также в плане быстродействия скриптовые языки можно разделить на языки динамического разбора (sh) и предварительно компилируемые (perl). Языки динамического разбора считывают инструкции из файла программы минимально требующимися блоками, и исполняют эти блоки, не читая дальнейший код. Предкомпилируемые языки транслируют всю программу в байт-код и затем исполняют его. Некоторые скриптовые языки имеют возможность компиляции программы «на лету» в машинный код (т. н. JIT-компиляция).