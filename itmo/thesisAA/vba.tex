\subsubsection{Visual Basic for Applications}

{\it Visual Basic for Applications} -- разработка компании {\it Microsoft}, совмещающая в себе реализацию языка программирования {\it Visual Basic} и соответствующую среду разработки. {\it VBA} является интерпретируемым языком: исходный код программы компилируется в промежуточный {\it P-код}, который впоследствии выполняется виртуальной машиной, являющейся частью основного приложения. {\it VBA} построен на технологии {\it COM}, благодаря чему {\it VBA-}код может использовать все доступные в операционной системе {\it COM}-объекты и компоненты {\it ActiveX}. Интегрированная среда разработки предоставляет удобный редактор кода с подсветкой синтаксиса и системой автодополнения кода, набор инструментов для удобной работы с проектами и файлами исходного кода, а также отладчик. Одно из важных достоинств {\it VBA} -- сравнительная лёгкость освоения.

Изначально {\it VBA} был встроен в семейство продуктов {\it Microsoft Office}. Из кода макроса на {\it VBA} пользователь может получить доступ ко всем функциям, доступным через графический интерфейс пользователя, а также к многочисленным функциям операционной системы и компонентов сторонних разработчиков.

Помимо {\it Microsoft Office} {\it VBA} встроен во многие программные продукты других разработчиков, к примеру,  {\it AutoCAD}, {\it SolidWorks}, {\it CorelDRAW}, {\it WordPerfect} и {\it ESRI ArcGIS}. Более того, {\it VBA} можно использовать при разработке любого {\it Windows}-приложения для внедрения в него возможности автоматизации. 

В настоящее время технология {\it VBA} считается устаревшей и более не поддерживается. В качестве замены {\it Microsoft} предлагает технологии {\it VSTA}~\cite{vsta-website} и {\it VSTO}~\cite{vsto-website}, речь о которых пойдёт в следующих разделах.